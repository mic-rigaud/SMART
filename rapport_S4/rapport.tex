\documentclass[a4paper, 11pt, oneside, oldfontcommands]{memoir}

%%%%% Packages %%%%%
\usepackage{lmodern}
\usepackage{palatino}
\usepackage[T1]{fontenc}
\usepackage[utf8]{inputenc}
\usepackage[french]{babel}


%%%%%%%%%%%%%%%%%%%%  PACKAGE SECONDAIRE

%\usepackage{amstext,amsmath,amssymb,amsfonts} % package math
%\usepackage{multirow,colortbl}	% to use multirow and ?
%\usepackage{xspace,varioref}
\usepackage[linktoc=all, hidelinks]{hyperref}			% permet d'utiliser les liens hyper textes
\usepackage{float}				% permet d ajouter d autre fonction au floatant
\usepackage{wrapfig}			% permet d avoir des image avec texte coulant a cote
%\usepackage{fancyhdr}			% permet d inserer des choses en haut et en bas de chaque page
\usepackage{microtype}			% permet d ameliorer l apparence du texte
\usepackage[explicit]{titlesec}	% permet de modifier les titres
\usepackage{graphicx}			% permet d utiliser les graphiques
\graphicspath{{./images/}}		% to say where are image
%\usepackage{eso-pic} 			% to put figure in the background
\usepackage[svgnames]{xcolor}	% permet d avoir plus de 300 couleur predefini
%\usepackage{array}				% permet d ajouter des option dans les tableaux
%\usepackage{listings}			% permet d ajouter des ligne de code
%\usepackage{tikz}				% to draw figure
%\usepackage{appendix}			% permet de faire les index
%\usepackage{makeidx}			% permet de creer les index
%\usepackage{fancyvrb}			% to use Verbatim
%\usepackage{framed}				% permet de faire des environnement cadre
%\usepackage{fancybox}			% permet de realiser les cadres
\usepackage{titletoc}			% permet de modifier les titres
\usepackage{caption}
\usepackage[a4paper, top=2cm, bottom=2cm]{geometry}
\usepackage{frbib}                      %permet d avoir une biblio francaise
\usepackage[babel=true]{csquotes}
\usepackage{eurosym}
\usepackage{frbib}
\usepackage[babel=true]{csquotes}
\usepackage[final]{pdfpages} 
\usepackage{listings}

\usepackage{graphicx}
\RequirePackage{pageGardeEnsta}	% permet d avoir la page de garde ensta

%\setcounter{secnumdepth}{2}		% permet d'augmenter la numerotation
%\setcounter{tocdepth}{2}		% permet d'augmenter la numerotation

%%%%%%%%%%%%%%%%%%  DEFINITION DES BOITES
\newcounter{rem}[chapter]

\newcommand{\remarque}[1]{\stepcounter{rem}\noindent\fcolorbox{OliveDrab}{white}{\parbox{\textwidth}{\textcolor{OliveDrab}{
\textbf{Remarque~\thechapter.\therem~:}}\\#1}}}

\newcounter{th}[chapter]

\newcommand{\theoreme}[2]{\noindent\fcolorbox{FireBrick}{white}{\stepcounter{th}
\parbox{\textwidth}{\textbf{\textcolor{FireBrick}{Théorème~\thechapter.\theth~:}}{\hfill \textit{#1}}\\#2}}}

\newcommand{\attention}[1]{\noindent\fcolorbox{white}{white}{\parbox{\textwidth}{\textcolor{FireBrick}{
\textbf{Attention !}}\\\textit{#1}\\}}}



%%%% Boite console:

% Definition de couleur supplementaire
\definecolor{colString}{rgb}{0.6,0.1,0.1}
 
% Definition du langage
\lstdefinelanguage{LangageConsole}{%
    morekeywords={%
        tim,% mot-clé ligne''
    },
    morestring=[b]",
    morecomment=[l]{./},
    morecomment=[s]{/*}{*/},
}
 
% Definition du style
\lstdefinestyle{styleLangage}{%
    language        = LangageConsole,%
    basicstyle      = \bf\footnotesize\ttfamily\color{white},% ecriture standard
    identifierstyle = \color{white},%
    commentstyle    = \color{red},%
    keywordstyle    = \color{green},%
    stringstyle     = \color{colString},%
    extendedchars   = true,% permet d'avoir des accents dans le code
    tabsize         = 2,%
    showspaces      = false,%
    showstringspaces = false,%
    numbers=left,%
    numberstyle=\tiny\ttfamily\color{black},%
    breaklines=true,%
    breakautoindent=true,%
        backgroundcolor=\color{black},%
}

\lstdefinestyle{shared}
{
    numbers=left,
    numbersep=1em,
    numberstyle=\tiny\color{red}\noaccsupp,
    frame=single,
    framesep=\fboxsep,
    framerule=\fboxrule,
    rulecolor=\color{red},
    xleftmargin=\dimexpr\fboxsep+\fboxrule\relax,
    xrightmargin=\dimexpr\fboxsep+\fboxrule\relax,
    breaklines=true,
    tabsize=2,
    columns=flexible,
}

\lstdefinestyle{python}
{
    style=shared,
    language={Python},
    %alsolanguage={[Sharp]C},
    basicstyle=\small\tt\color{black},
    keywordstyle=\color{blue},
    commentstyle=\color[rgb]{0.13,0.54,0.13},
    backgroundcolor=\color{white},
    morekeywords={
        Console,
        WriteLine,
        int,
    },
}
 
\lstset{%
    style = styleLangage%
}

\lstnewenvironment{python}
{\lstset{style=python}}
{}


%%%%%%%%%%%%%%%%%%%%%%%%%%%%%%%%%%%%%%%%%%%%%%%%%%%%%%%%%%%%%%%%%%%%%%%%%


%% INDEX %%%%%%%%%%%%%%%%%%%%%%%%%%%%%%%%%%%%%%%%%%%%%%%%%%%%
\makeindex

%%%%% Useful macros %%%%%
\newcommand{\latinloc}[1]{\ifx\undefined\lncs\relax\emph{#1}\else\textrm{#1}\fi\xspace}
\newcommand{\etc}{\latinloc{etc}}
\newcommand{\eg}{\latinloc{e.g.}}
\newcommand{\ie}{\latinloc{i.e.}}
\newcommand{\cad}{c'est-à-dire }
\newcommand{\st}{\ensuremath{\text{\xspace s.t.\xspace}}}
\newcommand{\rpi}{Raspberry Pi }

%%%% Definition des couleur %%%%

\newcommand\couleurb[1]{\textcolor{SteelBlue}{#1}}
\newcommand\couleurr[1]{\textcolor{DarkRed}{#1}}


%% number page style style %%%%%%%%%%%%%%%%%%%%%%%%%%%%%%%%%%%%%%%%%%%%%%%%%%%%%%

\pagestyle{plain}
%\pagestyle{empty}
%\pagestyle{headings}
%\pagestyle{myheadings}



%% chapters style %%%%%%%%%%%%%%%%%%%%%%%%%%%%%%%%%%%%%%%%%%%%%%%%%%%%%%
%% You may try several styles (see more in the memoir manual).

%\chapterstyle{veelo}
%\chapterstyle{chappell}
%\chapterstyle{ell}
%\chapterstyle{ger}
%\chapterstyle{pedersen}
%\chapterstyle{verville}
\chapterstyle{madsen}
%\chapterstyle{thatcher}


%%%%% Report Title %%%%%
%\title{Détecteur de drone par radio-goniométrie}
\title{System with Multi Antennas to Reorient a Target}
\author{D'Acremont Antoine\\Cotten Guillaume\\Legay Kevin\\Kennan Aya\\Shehade Mohammed\\Rigaud Michaël}
\date{\today}
%\doctype{Rapport}
\promo{promo 2017}
\version{2.0}
\etablissement{\textsc{Ensta} Bretagne\\2, rue François Verny\\
  29806 \textsc{Brest} cedex\\\textsc{France}\\Tel +33 (0)2 98 34 88 00\\ \url{www.ensta-bretagne.fr}}
\logoCentre{\includegraphics[height=4.2cm]{doppler}}
\logoBasGauche{\includegraphics[height=4.2cm]{drones-quadricoptere}}
\logoBasDroit{\includegraphics[height=4.2cm]{maps}}
\logoBas{\includegraphics[height=4.2cm]{logo_ENSTA_Bretagne_Vertical_CMJN}}
\logoSmart{\includegraphics[height=4.2cm]{Smart.png}}


%%%%%%%%%%%%%%%%%% DEBUT DU DOCUMENT
\begin{document}

\maketitle
\thispagestyle{empty}
\newpage

\tableofcontents


%%%%%%%%%%%%%%%%% INTRODUCTION


\chapter*{Remerciements}
\addcontentsline{toc}{chapter}{Remerciements}

Avant de commencer la présentation de notre travail, nous profitons de l’occasion pour adresser nos remerciements à toutes les personnes qui ont contribué de près ou de loin à la réalisation de ce projet.

Nous tenons à exprimer nos vifs remerciements pour notre respectueux professeur, M. Mansour Ali, d’avoir accepté de nous encadrer, suivre notre travail, nous diriger, afin que nous puissions mener ce projet à terme, ainsi que pour son soutien, ses remarques pertinentes et son encouragement.

Nos remerciements vont aussi à M. Le Chenadec Gilles , qui nous a accompagné de près durant tout ce travail, pour sa disponibilité, pour la confiance qu’il a su nous accorder et les conseils précieux qu’il nous a prodigués tout au long de la réalisation de ce projet.

Nos remerciements vont aussi à tous les professeurs, enseignants et toutes les personnes qui nous ont soutenus jusqu’au bout, et qui n’ont pas cessé de nous donner des conseils très importants en signe de reconnaissance. Nous souhaitons que le travail réalisé soit à la hauteur de leurs espérances ainsi qu’aux attentes de notre encadrant.


\begin{figure}[!h]
  \centering
  \includegraphics[width=0.3\textwidth]{merci}
\end{figure}

%%% Local Variables: 
%%% mode: latex
%%% TeX-master: "../rapport"
%%% End: 


\chapter*{Introduction}
\addcontentsline{toc}{chapter}{Introduction}


\newpage	  
%%%%%%%%%%%%%%%%%%%%%%%%

% \part{Etat de l'art \\(Semestre 3)}

% 
%\chapter{État de l'art des technologies}

\chapter{Présentation du contexte}

Dans le domaine de la détection de drones, après recherche littéraires et numérique, nous en avons conclu qu'il existait plusieurs types de détection: par acoustique, par méthodes optiques et par radiogoniométrie.
Ces méthodes possèdent chacunes leurs avantages et leurs inconvéniants que nous allons spécifier ci-dessous.


\section{Acoustique}

Plusieurs entreprises proposent des outils de détection des drones. Ces derniers se présentent sous forme de boîtiers reliés à des micros, positionnés en hauteur: c'est par le son de leurs hélices que les drones sont repérés, dans un rayon d'une centaine de mètres. Une alerte est alors envoyée sur un ordinateur ou par un SMS. Avantage: le système ne s'occupe pas des ondes, et peut détecter les drones autopilotés (voir plus bas). Problème: le bruit de fond doit être inférieur à un certain seuil, ce qui le rend difficilement utilisable en milieu urbain. De plus, pour des raisons d'échos, la multiplication de récepteurs est nécessaire afin de pouvoir filtrer le signal. Enfin, il est nécessaire de disposer préalablement d'une base de données des signatures acoustiques des différents drones qui peuvent émettre sur un domaine de fréquences acoustiques larges.

Cependant, ce système présente des failles. En effet, il est assez simple pour un drone de parer ce système de détection. Par la simple émission d'une onde sonore couvrant sa propre signature acoustique, un drone passerait totalement inaperçu.

Certains systèmes utilisent aussi une analyse fréquentielle poussée du signal afin de détecter les moteurs en fonction de leurs fréquences de fonctionnement.

Au-delà de cet aspect, il présente un avantage et des plus importants, son coût. En effet, un tel système est très économique à produire. Actuellement diverses solutions actives comme passives sont déjà présentes sur le marché. Ces solutions sont orientées vers une utilisation domestique et non professionnelle pour les raisons évoquées précédemment. Leur prix se situe aux alentours de 100 dollars pour un modèle classique, mais la multiplication des solutions tant à réduire le prix d'un tel système. 

\section{Optique}

Une caméra normale a besoin de lumière pour produire une image, une caméra thermique (ou infrarouge) peut capter de très faibles différences de température et les convertir en une excellente image thermique sur laquelle les plus petits détails sont visibles. Contrairement à d'autres technologies, comme l'amplification de lumière qui nécessite une petite quantité de lumière pour produire une image, l'imagerie thermique permet de voir dans l'obscurité totale. Elle ne nécessite aucune source de lumière.

Depuis qu'il est possible de produire une image lisible dans l'obscurité totale, la technologie de l'imagerie thermique permet de voir et de cibler les forces ennemies dans la nuit la plus noire. Les caméras thermiques voient à travers la brume, la pluie et la neige. Elles voient aussi à travers la fumée, ce qui était particulièrement intéressant pour l'armée.\cite{optique}

En mode passif, \emph{des caméras thermiques d'observation savent repérer un drone de 50 cm d'envergure à une distance d'environ 1 km, de jour comme de nuit} . Lorsqu'un drone entre dans son champ de vision, des algorithmes identifient son image. La forme, la couleur et la géométrie de l'objet permettent de distinguer le drone d'éventuels oiseaux et lancer une alerte, à condition qu'il n'y ait pas d'obstacle entre la caméra et lui.

En mode actif, on peut éclairer une scène à $360^{\circ}$ avec un laser. \emph{Les photons, les particules de lumière, se réfléchissent sur l'appareil, le signal est récupéré et analysé.} D'une portée similaire à celle de la caméra, le laser a l'avantage de \emph{décamoufler} (observation à travers brouillard, pluie ou filet de camouflage), de livrer la distance précise de l'objet, et de le reconstituer en imagerie 3D.Une fois le drone suffisamment proche, une caméra \textit{classique} avec un opérateur humain peuvent prendre le relai pour vérifier visuellement la nature de l'intrus et éventuellement passer à la phase de neutralisation.


\section{Radar}
Le radar (de l'anglais RAdio Detection And Ranging) est un système qui utilise les ondes électromagnétiques pour détecter la présence d'objets. Le radar émet des ondes, elles rebondissent sur les objets rencontrés et il est possible de mesurer leur distance, la direction, l'altitude ainsi que la vitesse en analysant le signal renvoyé. Les modèles Doppler peuvent ainsi détecter les objets en mouvement : avion, hélicoptère et certains modèles de drones, même « légers ». C'est le cas du radar Squire de Thales Air Systems. 

~\\

\includegraphics[width=\textwidth]{radar}
\captionof{figure}{Le radar portable Squire de Thales Air Systems}

Il existe néanmoins certains drones construits en carbone pouvant être perméables à certaines ondes radars et ainsi indétectable par cette technologie. Cependant le "radar passif", radar exploitant les variations d'ondes électromagnétiques en milieu urbain, telles que les ondes de la TNT, pourrait être exploité en milieu urbain.



\section{Radiogoniométrie}

Parmis les méthodes pour détecter un drone on peut citer la radiogoniométrie. Le principe de la radiogoniométrie est de mesurer la direction d'arrivée d'une onde électromagnétique polarisée incidente sur un réseau de capteur, par rapport à une direction de référence. Les radio-goniomètres sont donc des détecteurs passifs. 

La radiogoniométrie possède de nombreuses applications. Cependant, en interception, la radiogoniométrie permet de localiser un émetteur inconnu soit en employant plusieur récepteurs en des positions différentes, soit par calcul en fonction de la cinématique preopre du récepteur. 

On distingue deux types de goniomètres: les goniomètres à une dimension qui n'estiment que le gisement ou l'azimut, et les goniomètres à deux dimensions qui estiment le gisement ou azimut ainsi que l'élévation. 


Dans le cas d'une détection de drones, le radio-goniomètre réalise une écoute de l'environnement avec un balayage de fréquences. Lorsque le drone émettra avec la personne qui le guide on pourra ainsi le localiser précisément.

Seulement, la radiogoniométrie a des failles. En effet, il existe sur le marché des drones auto-pilotés qui n'émettent pas car ils chargent avant le début de leur vol leurs trajectoires. Ainsi il n'y a pas de communication avec un quelconque utilisateur, et donc il n'y a aucun signal émis. Il est donc impossible de les localiser à l'aide de cette technique.

Mais cette technique possède aussi ses avantages. C'est une technique passive et donc indécelable. C'est d'ailleurs pour cela que c'est une technique très utilisée dans la guerre électronique. 




\section{Synthèse}

Ainsi, la meilleure solution serait de réaliser un détecteur à base de ces trois modes de détection. C'est d'ailleurs pourquoi les produits les plus performants existant sur le marché utilisent un mélange de ces trois technologies. On peut notamment citer le cas du système drone-detector \cite{dronedetector}.

Néanmoins nous avons choisi pour ce projet de nous concentrer, dans un premier temps, sur une détection uniquement à base de radiogoniométrie.

% Ici il va falloir préciser plusieurs choses sur pourquoi ce choix. Notamment en précisant qu'on suppose que les drones respectent la réglementation, etc... 



%%% Local Variables: 
%%% mode: latex
%%% TeX-master: "rapport_analyse"
%%% End: 

% \chapter{Analyse fonctionnelle}

\section{Interview}

Après notre interview avec notre encadrant Ali Mansour, nous avons réalisé un tableau des spécifications suivantes:

\includegraphics[width=0.8\textwidth]{interview}


\section{Tableau des spécifications}
En prenant en compte les recommandations de notre encadrant, et les recherches que nous avons réalisées, nous avons établi les contraintes et les spécifications suivantes:

%\includegraphics[width=\textwidth]{tableauSpe}
\includepdf{./images/tableauSpe.pdf}

%Compte tenu des recherches que nous avons réalisées, nous avons établi l'étude fonctionnelle suivante.

%De la synthèse de ce tableau découle le diagramme Pieuvre et les SADT suivant.

\section{Diagramme pieuvre}
~\\
~\\
~\\
~\\
~\\
~\\
\hspace{-2cm}
\includegraphics[width=1.18\textwidth]{Diagramme_pieuvre.pdf}
\captionof{figure}{Diagramme pieuvre}



\section{SADT}

\includegraphics[width=\textwidth]{SADT_A-0.pdf}
\captionof{figure}{SADT A-0}
\includegraphics[width=\textwidth]{SADT_A0.pdf}
\captionof{figure}{SADT A0}

\newpage
\parindent=15pt

Comme on peut le voir sur le SADT A0, nous avons découpé notre objectif en trois parties.

Dans un premier temps il faut capter les signaux. Pour cela il faut réaliser un balayage sur le radiogoniomètre pour détecter les bons signaux.

Ensuite, il faut analyser les signaux reçus pour s'assurer que nous sommes bien en présence d'un drone.

Enfin, il faut récupérer les données des radiogoniomètres pour déterminer la position du drone.


\section{FAST}

\hspace{-1.5cm}
\includegraphics[width=1.2\textwidth]{FAST.pdf}
\captionof{figure}{Diagramme FAST}


\section{Diagramme 3 axes}

\hspace{-1.5cm}
\includegraphics[width=1.2\textwidth]{3axes.pdf}
\captionof{figure}{Diagramme 3 axes}
\parindent=15pt
~\\

Le diagramme 3 axes ci-dessus présente les étapes clefs du traitement du problème. En effet, la
détection d’un drone nécessite de repérer une perturbation dans la bande de fréquence que l’on écoute,
de détecter la direction de laquelle elle provient et enfin de regrouper les données pour, à partir des
directions, obtenir la position.

\section{Fonctionnement de notre système}

Nous avons donc imaginé positionner plusieurs radiogoniomètres, chaque appareil indiquerait la direction du drone par rapport à sa position. Chacun d'eux serait connecté à un ordinateur central qui analyserai chacune des positions données par les radiogoniomètres et en déduirait la position du drone dans l'espace.

~\\

\includegraphics[width=0.8\textwidth]{SMART_logic}
\captionof{figure}{Schéma Logique du système}
\parindent=15pt



%%% Local Variables: 
%%% mode: latex
%%% TeX-master: "rapport_analyse"
%%% End: 

% \chapter{Radio-goniométrie}

\section{Principe}

\subsection{Système Doppler}

\subsection{Système TDOA}

\subsection{Système Homing}








%%% Local Variables: 
%%% mode: latex
%%% TeX-master: "rapport_analyse"
%%% End:

% \chapter{État de l'art}


\section{Antennes}

\subsubsection{Principes généraux}

	Les goniomètres utilisent les ondes radioélectriques pour pouvoir localiser la direction d'une source d'émissions. Chaque type de goniomètre utilisera une ou plusieurs antennes pour pouvoir analyser les caractéristiques de l'onde reçue. 
	Le fonctionnement de ces antennes est décrit par les lois de l'électromagnétisme. Chaque onde électromagnétique possède une composante électrique et une composante magnétique. La composante ou champ électrique de cette onde fera apparaitre des variations de potentiel dans l'antenne dont l'amplitude et la fréquence seront directement liés à l'onde qui les a généré. Leur analyse permettra de récupérer les caractéristiques de l'onde reçue et d'en extraire les informations pertinentes pour le système équipé de l'antenne.
	
\subsection{Antennes en Radiogoniométrie}

	En radiogoniométrie il est possible de travailler avec plusieurs types d'antennes. La méthode la plus simple pour déterminer la direction d'une onde sera d'utiliser une antenne à ouverture dite faible et de la faire pivoter pour pouvoir déterminer la direction du maximum d'émission. Une antenne circulaire ou rectangulaire pourra convenir. Parfois le type de goniomètre utilisé déterminera le choix de l'antenne. Par exemple dans le cas d'un radiogoniomètre de type Watson-Watt plusieurs solutions sont envisageables: 
	
\begin{itemize}

\item l'utilisation de deux antennes circulaires ou rectangulaires.

\item L'utilisation d'une antenne dite "Adcock" qui est une combinaison d'antennes monopoles ou dipolaires. 

\end{itemize} 

	Dans le cadre de notre projet, nos recherches nous ont conduit à choisir un goniomètre Doppler. Ce type de goniomètre utilise au minimum quatres antennes monopoles ou dipolaires disposées en croix autour d'une antennes de référence omnidirectionnelle (une antenne monopole est souvent utilisée). Pour un nombre d'antennes supérieur celles-ci seront disposées en cercle à intervalle régulier autour de l'antenne de référence.
	En théorie deux antennes pourraient suffire. Si on parvenait à mettre en rotation une antenne omnidirectionelle autour de l'antenne de référence suffisamment rapidement le goniomètre Doppler fonctionnerait. Il est toutefois beaucoup plus simple d'utiliser un ensemble d'antennes disposées en cercle et "d'écouter" successivement chaque antenne à l'aide d'un commutateur pour simuler cette rotation.
	
\subsection{Système d'antennes retenu}

	Le goniomètre Montréal possède cinq antennes, quatre disposées en croix et une antenne centrale connectées au système électronique de traitement. Les antennes sont reliées à un commutateur permettant de sélectionner successivement les antennes de la croix. Le système est dimensionné autour de trois critères : 
	
\begin{itemize}

\item La bande de fréquence surveillée par les antennes : On utilisera ici des antennes monopoles (quart d'onde) adaptées au 2,4GHz

\item Le rayon d'écartement des antennes (distance entre les quatres antennes de la croix et l'antenne de référence) : Si ce rayon est trop faible les écarte de fréquence seront plus difficiles à remarquer et le bruit électromagnétique peut être plus gênant lors de la mesure.


\item La vitesse de commutation des antennes.

\end{itemize}

Deux des paramètres sont fixés à l'installation du dispositif. Les formules suivantes permettent de déterminer le paramètre manquant.

\begin{equation}

dF = (w*r*fc)/c

\end{equation}

et

\begin{equation}

fr = dF x 1879.8/R x fc

\end{equation}




%%% Local Variables: 
%%% mode: latex
%%% TeX-master: "rapport_analyse"
%%% End:

% \chapter{Le Montréal 3V2}
\label{montreal}

Nous allons ici présenter la solution sur laquelle nous nous appuyons pour réaliser notre propre radiogoniomètre à effet Doppler, le Montréal 3V2.
Pour réaliser cette documentation nous nous sommes appuyé sur la documentation trouvé sur le site f1lvt \cite{montreal}

\section{Évolution du Montréal}

\includegraphics[width=\textwidth]{evolution}
\captionof{figure}{Evolution du Montréal}

\section{Avantages du Montréal 3v2}

\begin{center}
  \includegraphics[width=0.5\textwidth]{montreal}
  \captionof{figure}{Photographie prise du Montréal 3v2}
\end{center}

\parindent=15pt
Le Montréal 3v2 sert principalement à l'FNRASEC\footnote{Fédération Nationale des Radioamateurs au service de la Sécurité Civile, agrée de sécurité civile} et aux chasseurs d'onde amateurs. Ce radiogoniomètre est utilisé pour la détection de balise de détresse de 406MHz.

%Parmi ses avantages, on peut noter qu'il est facile à construire, son prix , et il est simple d'utilisation. 

Un des intérêts majeurs du Montréal 3-V2, c'est sa capacité de localiser des signaux très courts, son prix de revient est très raisonnable,son traitement très rapide et la mise en mémoire automatique du dernier relevé. On peu aussi noter qu'il est simple d'utilisation grâce a son affichage à 36LED disposé en cercle et qui indique la direction. De plus une LED centrale est indique le fonctionnement; verte la direction affichée est bonne, rouge le signal est insuffisant, la direction reste alors figée dans la dernière bonne direction reçue.

%on peut noter son affichage à 36 LED qui indique la direction de manière clair et efficace, et un réglage facilité par son écran LCD ou encore son filtre à capa commutée à très faible largeur de bande (0,5 Hz).

\section{Caractéristiques}

Le Montréal 3v2 est un radiogoniomètre à effet Doppler, il possède donc toutes les caractéristiques associé a ce type de radiogoniomètre.
~\\

\begin{tabular}{ l l l}
Fréquences & distance & moyenne portée\\
 & gamme & 50MHz-1.3GHz\\
 & démodulation & FM\\
Affichage & LED & 36LED\\
& écran & LCD en 2 lignes\\
Filtre & capa & très faible largeur de bande (0.5Hz)\\
Coût & & estimé à 50\euro \\
\end{tabular}


\section{Fonctionnement}
La partie centrale contient les circuits d'amplification et de commutation. Les 4 brins verticaux (les brins actifs) se fixent par BNC.

Les antennes sont alimentées de façon séquentielle pour imiter une antenne en rotation. Une fois que les antennes ont capté les ondes provenant du drone, il faut faire une démodulation et enlever tous les bruits.


Un système à LED permet de visualiser la composante continue qui passe dans les antennes. A partir du boîtier Doppler et de son menu de test, on peut ainsi vérifier individuellement chaque antenne. Ceci permet soit de faire fonctionner le système Doppler avec une antenne sur 4 %(fonctionnement conforme à la théorie avec une seule antenne tournante)
, soit avec 3 antennes sur 4. %(ce qui inverse le signal Doppler à 500 Hz ; mais ça fonctionne aussi bien voire mieux).


Trois microcontrôleurs Pics sont utilisés un 16F628A pour l'affichage, un 16F877A pour le circuit principal et un 12F675 comme diviseur de fréquence.

Ce Doppler est la version la plus récente et la plus performante de la série. Il commute les antennes et il affiche la direction mesurée sur la boussole à 36 LED. 


\section{Schéma bloc}

\includegraphics[width=\textwidth]{schemaBloc}
\captionof{figure}{Schéma bloc du Montréal 3v2}

\section{Liste des composants}
Voici la liste des composants pour la construction du Montréal 3v2:

\begin{tabular}{ l l l}

IC30&          LM386N-4&                  Ampli BF\\
IC50&          MAX267BCNG&          Filtre\\
IC51& PIC  12F675-I/P& PIC \\
IC52&          74HC4051N&               Filtre\\
IC53&          MAX492CPA &            Ampli Op\\
IC70& PIC  18F4520-I/P& PIC\\
VR20 &       7805 TO-220  &            Régulateur\\
X70&           20 MHz  HC49&           Quartz\\
D50&           1N5819       &                Diode Schottky\\
LCD20&      LCD 2X16,&                 Afficheur 2 lignes de 16 car.\\
IC1& PIC16F628A-I/P& PIC\\
LED1 - LED36& ø3mm, Rouge et/ou Vert&\\
LED37&                        3 ou 5mm Bicolore Rouge/Verte &\\
FB1 - FB8&                   Ferrites\footnote{+ composants passifs : Résistances, Condensateurs.}&\\
IC100&        = MAX232ACPE&        en option\\
Q100 &        = 2N2222 TO-92&\\
\end{tabular}



%%% Local Variables: 
%%% mode: latex
%%% TeX-master: "rapport_analyse"
%%% End: 


%
\chapter{Contexte}

\section{Nature du besoin}

Les drones sont de plus en plus présents dans le monde moderne est font maintenant partie intégrante du paysage urbains. Il est en effet possible d'acheter pour 50\euro~  un drone miniature dans n'importe quel rayon de jouet de grandes surfaces, ... %Mais son usage ne s'arrête pas au loisir, de grandes firmes américaines comme Amazon souhaite utiliser ces drones pour livrer leur produit
Mais son usage ne s'arrête pas au loisir puisque l'actualité a montré que l'intrusion de drones dans des sites sécurisés représentaient un risque de sécurité majeur. Le risque de sécurité que représentent ces drones peut aussi s'étendre à d'autres lieux, moins sensibles, mais ou leur intrusion peut avoir des conséquences désastreuses comme un aérodrome de campagne ou au dessus d'un terrain de sport pendant une compétition.


Notre projet, SMART (System with Multi Antennas to Reorient a Target), doit répondre a ce problème en permettant de détecter ces drones.  

\section{Etat d'avancement du projet}


L'état de l'art à permis de déterminer plusieurs méthodes pour détecter un drone aérien. Elles sont principalement acoustiques, optiques ou électromagnétiques. Compte tenu du budget et de la complexité des différents systèmes observés l'équipe a opté pour une solution entièrement électromagnétique. La solution envisagée est un système passif de radio-goniométrie qui réceptionne les ondes émises par le drone puis utilise l'effet Doppler pour obtenir la direction d'émission par rapport à un système d'antennes fixe. Un dispositif muni de deux systèmes d'antennes sera alors en mesure d'obtenir la position approximative du drone a détecter. 
La connaissance de la position du drone pourra servir au développement de systèmes de brouillage ou de piratage du drone pour le neutraliser définitivement.

%%% Local Variables: 
%%% mode: latex
%%% TeX-master: "../rapport"
%%% End: 

%\chapter{Ingénierie Système}


%%% Local Variables: 
%%% mode: latex
%%% TeX-master: "../rapport"
%%% End: 

% \part{Conception de Smart \\(Semestre 4)}

\part{Préparation}



\chapter{Point de Situation}
\label{chap:choix}

\section{Contexte}
La première partie du projet nous a permis de redéfinir le sujet et les attentes, mais également de réfléchir aux solutions technologiques que nous pourrions utiliser. Après avoir observé les solutions existantes et compte tenu du budget qui nous est imposé, la solution générale retenue sera donc la suivante: les ondes radio captées par les antennes, seront traitées par un filtre et un down-converter avant de passer par le radiogoniomètre et obtenir une position angulaire qui sera envoyée au serveur puis transmise à l’IHM (voir dans le chapitre \ref{chap:choix} la partie \ref{sec:phys}).

Cependant, de nombreuses zones d’ombres planent encore sur la réalisation technique, notamment car nous ne maitrisons pas le domaine. En effet, nous utiliserons des plans de carte électronique de radiogoniomètre amateur afin de réaliser notre solution technologique, car réaliser un traitement entièrement numérique ne semble pas réalisable dans le temps impartit. 

De plus, actuellement, nous ne nous sommes pas encore réellement penchés sur le traitement des informations ainsi que sur la structure du serveur et de l’IHM. Nous avons seulement convenu que le transfert des informations se fera via Arduino. L’un d’entre eux sera le serveur et dialoguera avec l’ensemble des autres Arduino positionnés sur les antennes.~\\


Actuellement, nous ne possédons aucune partie physique du projet et nous réalisons des listes de commandes afin de pouvoir commencer les tests unitaires le plus rapidement possible.

\section{Rappel de notre projet}

Suite à notre état de l'art, nous avons décidé de réaliser notre système de détection en installant un maillage de capteur qui se baseront sur le système du Montréal 3V2. Chaque capteur sera connecté à un \rpi 2 \footnote{La documentation technique du Raspberry PI est situé en annexe à la page \pageref{annexe:rpi}}. De plus, chaque \rpi communiquera avec un ordinateur central qui traitera les données pour les afficher sur une interface graphique. Les données qui seront transmises sont: le numéro du \rpi, la position du capteur, et le gisement du drone par rapport au capteur. Enfin, l'ordinateur central communiquera avec une application android qui notifiera le client de la présence d'un drone comme on peut le voir sur la figure \ref{fig:inst}.
~\\

\begin{figure}[!h]
  \centering
  \includegraphics[width=\textwidth]{installation}
  \caption{Installation de Smart}
  \label{fig:inst}
\end{figure}


\section{Architecture Fonctionnelle}

\section{Architecture physique}
\label{sec:phys}


L'architecture physique du système est présenté à la figure \ref{fig:arch_phys}.

\begin{figure}[!h]
  \centering
  \includegraphics[width=\textwidth]{fonctionnement}
  \caption{Architecture Physique}
  \label{fig:arch_phys}
\end{figure}

\newpage
\section{Treillis de détection}

Pour répondre au besoin de détection et s’assurer d’un correct positionnement de la cible, la mise en
place d’une couverture de détection répondant à nos besoins était nécessaire. Les critères retenus
pour cette dernière sont les suivants :

\begin{itemize}
\item A l’intérieur de la zone de détection, la cible doit être en permanence sous la couverture de détection de 4 radiogoniomètres
\item Tenter une optimisation de la couverture afin d’éviter l’installation d’un trop grand
nombre de radiogoniomètre
\end{itemize}

Très peu de sujets similaires ont pu être trouvé bien que le problème soit récurrent dans de
nombreux projets.

Cependant, après plusieurs essais, le choix de treillis fut le suivant :

\begin{minipage}{0.45\linewidth}

  \centering
  \includegraphics[width=\textwidth]{treillis_explication}
  \captionof{figure}{Cercle de détection}
  ~\\
\end{minipage}
\begin{minipage}{0.45\linewidth}
  \begin{itemize}
  \item Dans les deux cas, les distances d et d’ représente la distance de couverture maximale d’une antenne pour une configuration de treillis particulière, distance au-delà de laquelle nous ne sommes pas sûr d’assurer la détection d’un drone.
  \item Chaque croix noire représente une antenne. Ces dernières formes ainsi la zone de détection, zone à l’intérieure de laquelle, le drone se doit d’être repéré.
  \end{itemize}
\end{minipage}


Initialement, nous avions prévu que les antennes radiogoniométrique assureraient la détection
jusqu’à son plus proche voisin. Cependant, cette configuration ne permet pas d’assurer qu’un drone
traversant la zone soit sous la couverture d’au moins quatre antennes en tout temps. Nous avons
donc choisi la configuration représentée par le cercle rouge, c’est-à-dire en rapprochant les antennes
les unes des autres. On obtient ainsi la couverture suivante :
~\\

\begin{minipage}{0.45\linewidth}
  \centering
  \includegraphics[width=\textwidth]{cercle}
  \captionof{figure}{Maillage de détection}
\end{minipage}
\begin{minipage}{0.45\linewidth}
  Dans cette configuration, les zones de plus faible couverture
sont situées sur les antennes elles même. En effet, au-dessus
de chaque antenne, la couverture n’est assurée que par
quatre d’entre elles. En dehors de celle-ci, la couverture est
assurée par cinq à six antennes.
\end{minipage}




%%% Local Variables: 
%%% mode: latex
%%% TeX-master: "../rapport"
%%% End: 

\chapter{Tests unitaires}

Suite au choix que nous avons réalisé dans la partie précédente, nous avons commandé notre matériel. Dès la réception de celui-ci nous avons effectué des tests unitaires pour vérifier leur bon fonctionnement. Voici la liste de l'ensemble des tests que nous avons réalisé.



%%%%%%%%%%%%%%%%%%%%%%%%%%%%%%%%%%%%%%%%%%%%%%%%%%%

\section{PIC}
\label{sec:pic}

Dans le schéma du Montréal 3v2 nous avons pu constater qu'il y avait 3 PIC programmés. Nous avons commandé les PIC programmés au près de l'entreprise F1LVT \cite{montreal}.


\begin{figure}[!h]
  \centering
  \includegraphics[width=\textwidth]{Test_unitaire/Pic/pic}
  \caption{3 PIC programmés}
  \label{fig:pic}
\end{figure}

Nous avons ensuite imaginé et réalisé des tests unitaires sur chacun des PIC pour vérifier qu'ils ont bien été programmés et qu'aucune erreur n'est apparu sur ce système de décision critique pour le système.

\subsection{PIC16F628A}
\label{sec:picled}

Ce PIC sert à réaliser l'affichage sur les LED. Pour tester ce PIC, nous avons réaliser le montage de la figure \ref{fig:picled}. On peut voir à la figure \ref{fig:schemapic} le schéma de montage du PIC sur le Montréal 3v2.

\begin{figure}[!h]
  \centering
  \includegraphics[width=\textwidth]{Test_unitaire/Pic/picled}
  \caption{Schéma electrique du test unitaire}
  \label{fig:picled}
\end{figure}


Pour tester ce composant, nous avons donc choisi de monter une partie des LED situés en sorti, de configurer le \og clock\fg{} sur un signal carré de fréquence 1 MHz, et de faire varier la fréquence de l'entrée \og data\fg{}.

Malheureusement nous n'avons pas pu observer de LED s'allumer pendant notre expérience.

\begin{figure}[!h]
  \centering
  \includegraphics[width=\textwidth]{Test_unitaire/Pic/schemapic}
  \caption{Schéma de montage du pic sur le Montréal 3v2}
  \label{fig:schemapic}
\end{figure}
\newpage
\subsection{PIC12F675 et PIC18F4520}
\label{sec:picclock}


Ces deux PIC ont également été testé avec le même procédé. Nous ne développerons pas plus ces tests car ils nous ont donné les même résultats.

\section{Filtre passe bande}
\label{sec:passe_bande}


Pour le filtre passe bande, nous souhaitions un filtre qui couperait tout ce qui se trouve en dehors de notre bande. Les tests ont montrés que ce filtre réagit plutôt bien quand le montage qui y est lié est adapté, ce qui est le cas, le circuit fonctionne bien avec une impédance de 50 Ohm.

Le test unitaire était simple on a branché le filtre sur un analyseur que envoyais un signal et recevait ce même message. Il est alors simple d’obtenir le comportement du filtre. Nous avons obtenu que le filtre diminuait très bien ce que ce trouve avant 2.4GHz mais plutôt mal ce qui vient après. Ceci n’est pas gênant car les bande entre 2.5Ghz et 5 GHz sont peu utilisées en France. 

\begin{figure}[h]
  \centering
  \includegraphics[width=0.8\textwidth]{oscillo1}
  \caption{Comportement du filtre}
  \label{fig:comportement}
\end{figure}

Sur la photo le curseur sur la courbe est à 2.45Ghz et le plat est un peu plus grand que la bande.

Nous avons mesuré deux paramètres supplémentaires, Le S11 et le S21 qui sont des paramètres permettant de mesurer la perte d’amplitude lié au composant. Le S11 est le coefficient de réflexion à l'entrée lorsque la sortie est adaptée. Dans l’idéal il vaut 0, il n’y a alors aucune réflexion et tout l’amplitude du signal sort du filtre, on obtient le S11 de la photo suivante.

On peut voir ici que le log du S11 est très faible entre 2.4 et 2.5 GHz ce qui indique un faible taux de réflexion et donc que le signal en entrée sera peu atténué.

\begin{figure}[h]
  \centering
  \includegraphics[width=0.8\textwidth]{oscillo2}
  \caption{S11 du filtre passe bande}
  \label{fig:filtre}
\end{figure}


Les deux plus grands pics vers le bas correspondent aux limites de la bande 2.4-2.5GHz.

Le S21 est le coefficient de transmission direct lorsque la sortie est adaptée, pour celui-ci le but est d’avoir ce nombre le plus proche de 1 et donc son logarithme le plus proche de zéro possible.

Lors du test, nous avions une perte d’environ 2dBm dans la bande de fréquence 2.4-2.5GHz, ce qui est relativement faible, le filtre ne risque donc pas d’occulter ce que l’on souhaite voir en atténuant trop fortement le signal utile.
\newpage
\section{VCO}



Le test du VCO est simple mais doit être bien fait car sans lui impossible d’obtenir la bonne fréquence de travail en entrée du radio-goniomètre.

Nous avons commencé par mesurer la fréquence libre, c’est-à-dire la fréquence renvoyée par le VCO lorqu'il est alimenté avec une tension nulle en entrée. La fréquence libre mesurée sur notre matériel était de 1.35 GHz donc plus importante que la fréquence indiqué par le constructeur (1.31Ghz). L'étape suivante consistait a mesurer la tension pour laquelle nous obtenions une fréquence de sortie de 1.9Ghz et nous avons obtenu environ 8V ce qui nous à permis de choisir le bon régulateur de tension pour la suite.
 Les régulateur sont calibrés, il est donc difficile d’en trouver un qui corresponde parfaitement mais nous avons pu obtenir un régulateur à 8.1V qui après test donnait en sortie du VCO une fréquence de 1.91GHz. La bande de fréquence a transférer étant de 100 MHz nous n’étions pas à 10 MHz prés et il aurait été difficile et couteux de trouver un meilleur moyen de le faire.


\begin{figure}[h]
  \centering
  \includegraphics[width=0.8\textwidth]{oscillo3}
  \caption{Fréquences généré par le VCO alimenté à 8.1V centrée autour de 1.9GHz d’une largeur d’environ 10MHz}
  \label{fig:freq}
\end{figure}
\newpage
\section{Down converter}
\label{sec:down_converter}



\begin{figure}[h]
  \centering
  \includegraphics[width=0.8\textwidth]{oscillo4}
  \caption{Montage de test de l’adaptateur}
  \label{fig:mont}
\end{figure}


Le test du down-converter a posé quelques problèmes. En effet le seul moyen de le tester est de le tester dans son cas d’utilisation pratique. Il est, en effet, nécessaire de l’alimenter et  ceci ne peut se faire sans le VCO, de même le bruit risque de gêner l’observation, il faut donc utiliser le filtre.


\begin{figure}[h]
  \centering
  \includegraphics[width=0.8\textwidth]{oscillo5}
  \caption{Photo du VCO et de son alimentation}
  \label{fig:photo}
\end{figure}



Nous avons rencontré des problèmes lors du test, premièrement, des problèmes de communication entre membre de l’équipe ont retardé le test d’une semaine, en effet il a fallu crée un montage avec les deux régulateurs de tension, celui pour l’alimentation et celui pour la tension en entrée. Il manquait cependant une information relative à l'orientation d'un composant sur les schémas fourni au membre de l'équipe chargé de la soudure et cela a entrainé une erreur de montage avec une augmentation des délais liés à la réalisation de ces filtres.

Deuxièmement, nous n’avions pas prévue les problèmes dus aux câbles. Il a fallu en effet cherché des câbles en n’étant pas certain que le câblage utilisé ne ferait pas griller le matériel.

Troisièmement, le professeur nous ayant aidé lors de la conception théorique du montage était en déplacement, il n’a donc pas pu nous aider lorsque des hésitations se sont fait sentir. Devant le prix du matériel et la possibilité de l'endommager, nous avons choisis d'attendre.





%%% Local Variables: 
%%% mode: latex
%%% TeX-master: "../rapport"
%%% End: 


\part{Réalisation}



\chapter{Radio-Goniometre}

 
\section{Module Montréal}

Le module Montréal comprends deux cartes électroniques : la carte principale qui est chargée de toutes les tâches relatives au traitement du signal reçu par les antennes et la carte destinée à l'affichage de la direction.

Les supports des cartes sont en résine époxy sur laquelle ont été apposée des pistes en cuivre en configuration mono-couche (les pistes ne sont présentes que sur une face de la plaque). Un étamage par dépôt chimique des piste a été réalisé pour les protéger de l'oxydation. L'ensemble de ces opération ont été effectuées à l'école avec l'aide de Mr. Gallou.
 
Le PIC18F4520 chargé de la majorité des traitements et le PIC12F675 ont été soudés sur la carte principale. Le PIC16F628A, chargé de l'affichage des LEDs a été soudé sur la carte d'affichage de direction. Le montage choisi pour ce dernier PIC permet de le désolidariser de la plaque afin d'effectuer les tests d'allumage des LEDs de direction.

L'ensemble des composants que nous avions a disposition (LEDs, condensateurs et résistances) ont été ensuite soudés sur les deux plaques. 

Par rapport au montage d'origine nous n'utiliseront pas le RS232. Toutefois, lors de la rédaction de ce rapport, il manque deux filtres sur la carte principale qui nous empêchent de valider son fonctionnement.


\section{Down converter}


\begin{wrapfigure}{r}{0.4\textwidth}
  
  \includegraphics[width=0.4\textwidth]{mixer}
  \caption{schéma de fonctionnement d'un mixer}
\end{wrapfigure}


Le radio-goniomètre Montréal 3v2 fonctionne à une fréquence de 500Mhz. Sans modification il est impossible de l’utiliser entre 2.4Ghz et 2.5 GHz, bande de fréquence utilisée par les drones que nous souhaitons détecter. Nous avons donc cherché un moyen d’adapter ce radio-goniomètre aux fréquences souhaitées.

Une solution applicable à notre système est l'utilisation d'un down-converter. Ce composant reçoit deux entrées, le signal dont on veut changer la fréquence(RF) et un signal de fréquence fixe Df(LO). Le down-converter diminue la fréquence du premier signal de celle du second. La sortie(IF) correspond au signal modifié. Son principe de fonctionnement est illustré à la figure \ref{fig:mix}.



\begin{figure}[h]
  \centering
  \includegraphics[width=\textwidth]{fonc_mixer}
  \caption{principe du fonctionnement d'un mixer}
  \label{fig:mix}
\end{figure}

On utilisera donc le down-converter pour abaisser la fréquence reçue par les antennes (2.4GHz) à la fréquence de travail du radio-goniomètre.

La fréquence du signal Df est donnée par un VCO (Voltage Controlled Oscillator ou oscillateur contrôlé en tension), le VCO reçoit en entrée une tension et donne en sortie une sinusoïdal de fréquence dépendante de la tension d’entrée. Le VCO étant très sensible, il est nécessaire de stabiliser la tension d’entrée et l’alimentation. On utilise donc un régulateur de tension qui amène une entrée stable. Le régulateur est un composant qui viendra limiter à une valeur seuil Vmax la tension qu'il reçoit en entrée si celle ci dépasse ce seuil. Dans le cas ou V < Vmax la sortie du régulateur sera égale à l'entrée.

Le VCO étant un composant actif et sensible aux variations dans sa tension d'alimentation, un second régulateur a été utilisé pour alimenter le VCO, toujours dans le but d’obtenir une fréquence stable ne variant pas pendant le processus. Il est en effet indispensable que cette fréquence reste fixe pour que l’effet doppler soit toujours visible et exploitable. Des fluctuations incontrôlées de la tension d'alimentation ou de la fréquence pourraient perturber la localisation.

Pour améliorer la mesure, il est utile de filtrer au maximum toutes les fréquences pouvant perturber la mesure et les différents bruits électromagnétiques. Ces phénomènes ont été atténués en positionnant en entrée du down-converter un filtre passe-bande qui permet de conserver uniquement la portion du spectre qui nous intéresse soit la bande située entre 2.4Ghz et 2.5Ghz.


\begin{figure}[h]
  \centering
  \includegraphics[width=0.5\textwidth]{down_converter}
  \caption{le down-converter et le filtre passe bande}
  \label{fig:down}
\end{figure}

A l’aide de ce montage on peut améliorer le signal en entrée pour qu'il puisse être traité correctement par le radio-goniomètre.

%%% Local Variables: 
%%% mode: latex
%%% TeX-master: "../rapport"
%%% End: 


\chapter{Interface Web}
\label{Logiciel}

Une foi le capteur terminé nous nous sommes penchés sur l'interface Homme/Machine. Dans une volonté de délivrer à l'utilisateur une interface agréable et lisible, nous avons décidé de proposer dans un premier temps une interface web.

\section{Analyse}
\label{sec:uml}

Dans un premier temps nous nous sommes penché sur une  phase d'analyse.

Dans la phase d’analyse, on cherche d’abord à bien comprendre et à décrire de façon précise les besoins des utilisateurs ou des clients concernant cette interface. Que souhaitent-ils faire avec le logiciel ? Quelles fonctionnalités veulent-ils ? Pour quel usage ? Comment l’action devrait-elle fonctionner ? C’est ce qu’on appelle \og l’analyse des besoins\fg{}. Après validation de notre compréhension du besoin, nous imaginons la solution. C’est la partie analyse de la solution.

Dans la phase de conception, on apporte plus de détails à la solution et on cherche à clarifier des aspects techniques, tels que l’installation des différentes parties logicielles à installer sur du matériel. Pour réaliser ces deux phases dans un projet informatique, nous utilisons des méthodes, des conventions et des notations. UML fait partie des notations les plus utilisées aujourd’hui. Pour faciliter à nos clients d’obtenir la direction des drones on a créé une interface web qui répond à leur besoin.

\subsection{Uml}

Pour décrire au mieux ce besoin, nous avons commencé par réaliser un cas d'utilisation de l'interface (figure \ref{fig:use_case}).
\begin{figure}[!h]
  \centering
  \includegraphics[width=\textwidth]{use_case}
  \caption{Cas d'utilisation de l'interface}
  \label{fig:use_case}
\end{figure}

\newpage
Ensuite, nous avons cherché a réaliser un diagramme de classe de notre interface. Pour cela nous avons défini 3 classes principales:

\begin{itemize}
\item index.php, qui réalise l'affichage dans un navigateur
\item serveur.py, qui récupère les données de chacun des radiogoniomètres
\item client.py, installé sur chaque radiogoniomètres il envoie les données des capteurs à travers un socket au serveur.
\end{itemize}

Le diagrammes de classe de la figure \ref{fig:class}, montre ce fonctionnement.

\begin{figure}[!h]
  \centering
  \includegraphics[width=\textwidth]{class_diagram}
  \caption{Diagramme de classe}
  \label{fig:class}
\end{figure}



\section{Conception}

\subsection{Client-Serveur}

Pour commencer nous avons cherché à donner une interface à nos \rpi pour communiquer, c'est l'interface Client/Serveur.
~\\

Cette interface a été codé en python. Nous avons choisi le python car c'est un langage souple et rapide. Nous avons réalisé la connection à l'aide d'un socket TCP/IP. De plus, le serveur se base sur du multi thread pour accepter plusieurs client. Le client se connecte donne son identifiant puis sa communication est placé dans un thread. Ainsi, on a un serveur qui peut accepter une infinité de client.
~\\


\begin{minipage}[h]{0.45\linewidth}

\begin{lstlisting}
./serveur.py
Serveur pret, en attente de requetes ...
Client RP1 connecte, adresse IP 127.0.0.1, port 51632.
RP1> led21
donc la direction 200
[...]
Client RP1 deconnecte.
[...]
Fin du serveur
\end{lstlisting}  
\end{minipage}\hfill
\begin{minipage}[h]{0.45\linewidth}
  
\begin{lstlisting}
./client.py
Connexion etablie avec le serveur.
Vous etes connecte. Envoyez vos messages.
la led allume: led21
[...]
Connexion interrompue.
\end{lstlisting}

\end{minipage}





\subsection{Web}


\subsection{Treillis de détection}

Pour répondre au besoin de détection et s’assurer d’un correct positionnement de la cible, la mise en
place d’une couverture de détection répondant à nos besoins était nécessaire. Les critères retenus
pour cette dernière sont les suivants :

\begin{itemize}
\item A l’intérieur de la zone de détection, la cible doit être en permanence sous la couverture de détection de 4 radiogoniomètres
\item Tenter une optimisation de la couverture afin d’éviter l’installation d’un trop grand
nombre de radiogoniomètre
\end{itemize}

Très peu de sujets similaires ont pu être trouvé bien que le problème soit récurrent dans de
nombreux projets.

Cependant, après plusieurs essais, le choix de treillis fut le suivant :

\begin{minipage}{0.45\linewidth}

  \centering
  \includegraphics[width=\textwidth]{treillis_explication}
  \caption{Cercle de détection}
  ~\\
\end{minipage}
\begin{minipage}{0.45\linewidth}
  \begin{itemize}
  \item Dans les deux cas, les distances d et d’ représente la distance de couverture maximale d’une antenne pour une configuration de treillis particulière, distance au-delà de laquelle nous ne sommes pas sûr d’assurer la détection d’un drone.
  \item Chaque croix noire représente une antenne. Ces dernières formes ainsi la zone de détection, zone à l’intérieure de laquelle, le drone se doit d’être repéré.
  \end{itemize}
\end{minipage}


Initialement, nous avions prévu que les antennes radiogoniométrique assureraient la détection
jusqu’à son plus proche voisin. Cependant, cette configuration ne permet pas d’assurer qu’un drone
traversant la zone soit sous la couverture d’au moins quatre antennes en tout temps. Nous avons
donc choisi la configuration représentée par le cercle rouge, c’est-à-dire en rapprochant les antennes
les unes des autres. On obtient ainsi la couverture suivante :
~\\

\begin{minipage}{0.45\linewidth}
  \centering
  \includegraphics[width=\textwidth]{cercle}
  \caption{Maillage de détection}
\end{minipage}
\begin{minipage}{0.45\linewidth}
  Dans cette configuration, les zones de plus faible couverture
sont situées sur les antennes elles même. En effet, au-dessus
de chaque antenne, la couverture n’est assurée que par
quatre d’entre elles. En dehors de celle-ci, la couverture est
assurée par cinq à six antennes.
\end{minipage}


\subsection{Triangulation}

Les différents radiogoniomètres nous donnant un gisement de la détection, nous pouvons donc
réaliser une triangulation de la position du drone lorsqu’un nombre suffisant d’antenne détecte le
drone. Bien qu’une première estimation de la position peut-être obtenu à partir de deux drones,
nous considérons que le drone doit être détecté par au moins quatre senseurs pour que la position
soit acceptable.
~\\

Cependant, avant de pouvoir réaliser toute triangulation, l’acquisition des points d’intersection entre
les droites de détection issue du gisement fourni par les différents radiogoniomètres est nécessaire.
Pour se faire, à partir des angles, nous formulons une équation de droite plan, passant par les
radiogoniomètres respectifs, et tentons de trouver une solution à chaque système composé de deux
droites.
~\\

\begin{wrapfigure}{r}{0.5\textwidth}

  \includegraphics[width=0.5\textwidth]{triangulation1}
  \caption{triangulation}
\end{wrapfigure}

  Suite à la résolution de ces différents systèmes, nous
obtenons une liste de différentes solutions, solutions ici
schématisé avec des points de couleur grise. Nous observons
que, du fait de la portée de détection, ainsi que la géométrie
de notre treillis, certains points sont incohérents ou très
imprécis.

~\\
Notre première approche fut donc de positionner le drone à
la moyenne de l’ensemble des positions solutions d’un des
systèmes précédent. Cependant, après simulation, il s’est
avéré que, de par l’imprécision relative des
radiogoniomètres (les angles sont donnée à 5\up{o} près), la moyenne de donnait qu’une idée toute relative de la position du drone et souvent loin de la vérité, et
cela à cause d’intersections multiples entre les droites de détection.
Pour corriger cela nous avons fait le choix d’utiliser la médiane afin de supprimer tous les résultats
incohérents. A partir de la médiane des résultats, nous appliquons un gabarit circulaire et retenons
tous les points d’intersection compris dans ce gabarit. Une moyenne est alors appliquée à l’ensemble
de ces résultats nous permettant d’obtenir un résultat plus cohérent et moins sensible aux erreurs.



\begin{figure}[!h]
  \centering
  \includegraphics[width=\textwidth]{interface}
  \caption{Interface Web}
  \label{fig:interface}
\end{figure}


%%% Local Variables: 
%%% mode: latex
%%% TeX-master: "../rapport"
%%% End: 





\chapter{Application Android}

\section{Présentation du SMART Comm Center}

	Le système SMART se base sur un ensemble de capteurs reliés à une station centrale dont les données sont accessibles depuis internet. Cela permet d'accéder aux paramètres du système à distance depuis un autre appareil relié à internet de préférence un ordinateur. Toutefois, le système, étant donné son coût réduit se destine à être utilisé au sein de structures de petites tailles ou le personnel en charge de la sécurité du site doit parfois se déplacer en fonction de ses autres obligations et ne peut être constamment en train de surveiller l'état du SMART depuis un ordinateur. 
	
	Pour pallier à ce manquement, il semblait intéressant de proposer une solution sur téléphone mobile qui permettrait d'avertir l'utilisateur final où qu'il se trouve. Deux options existent :
	~\\
	\begin{itemize}
	
	\item Une version mobile de l'interface web
	\item Une application dédiée	
		
	\end{itemize}
	~\\
	 Ces deux utiliseraient les APIs des différentes plateformes mobiles existantes (Windows phone, Android, iOS) pour avertir l'utilisateur en utilisant les fonctions vibreur ou la sonnerie du téléphone.

	 Toutefois, les capacités du site mobile sont assez limitées car de nombreux éléments de sécurité peuvent restreindre l'accès à certaines fonctionnalités du téléphone comme l'accès au vibreur ou la possibilité de s'exécuter en tache de fond. De plus ces restrictions varient en fonction de la plateforme mobile visée.

	
	 L'application mobile a donc l'avantage d'offrir plus de latitude au développeur et d'implémenter plus facilement différents moyens d'alerte pour l'utilisateur. Il faut cependant garder à l'esprit le fait que ce choix de développement implique de réaliser une application par système d'exploitation mobile existant.
	~\\
	Le choix final pour la version actuelle du système SMART à donc été celui de l'application mobile. Compte tenu des équipements dont nous disposions le système sur lequel l'application a été développée est Android. En effet, une grande partie du code source est sous licence GPL \footnote{GPL : "General Public License" régissant la distribution des logiciels libres} et le codage des applications se fait en Java dans sa version 1.7. De plus, ce système d'exploitation mobile représente en Janvier 2016 64 \% du parc mobile français.

\section{Fonctionnement de l'application}

\subsection{Vue d'ensemble}

	Le système SMART utilise un serveur web pour gérer l'affichage des donnés à destination de l'utilisateur final, il est en mesure de créer des connexions vers plusieurs appareils distants. L'application jouera le rôle de client et recevra du serveur les informations de position du drone en cas d'intrusion. Dans l'état actuel la réception des données de position par l'application se fait en mode "pull". Cela signifie que c'est l'utilisateur qui lance la demande d'information et le serveur répond ensuite à la requête.
	
	Un mode automatique, avec un rafraichissement régulier de l'information, a été codé et implémenté mais n'a pas été utilisé dans la version finale de l'application. Il est aussi possible de passer par un mode "push", où le serveur envoie l'information de lui même vers le client dès que celle-ci est mise à jour. Les raisons de ces choix technologiques seront détaillées par la suite.


\subsection{Choix du niveau d'API}

	La version actuelle de SMART Comm Center a été développée avec un niveau d'API Android minimum de 19 \footnote{L'API correspond à la version d'Android ciblée et détermine donc les fonctionnalités disponibles pour le développeur. le niveau 19 correspond à Android 4.4 KitKat.}. Le choix d'un niveau aussi élevé d'API a été déterminé par trois éléments importants : la gestion des "threads", des tâches asynchrones et des sockets. En effet depuis l'API 19 Google, qui édite et maintient le code d'Android, a modifié la façon dont les connexions réseau étaient gérées sous Android et le fonctionnement actuel qui sera détaillé par la suite nous convenait mieux. 
	
	Il faut cependant noter qu'un tel choix limite le nombre de smartphones qui seront en mesure de lancer l'application. L'API 22 par exemple représente seulement 34 \% du nombre total d'appareils Android activés. Dans un souci de faciliter la réalisation de l'application nous avons maintenu ce choix, sachant que l'ensemble des téléphones pouvant exécuter du code écrit avec un niveau d'API de 19 représente en Janvier 2016 75.6\% du nombre total des téléphones Android activés dans le monde. Ce choix n'est donc pas si restrictif au vu du nombre d'appareils touchés (plus d'un milliard).

\subsection{Achitecture détaillée}


\begin{figure}[h]
  \centering
  \includegraphics[width=0.5\textwidth]{projet_app}
  \caption{Diagramme de classe de l'application}
  \label{fig:down}
\end{figure}

	La structure globale de l'application est visible dans le diagramme ci-dessus. Elle regroupe quatre classe, trois liés aux différents écrans de l'application (XxxActivity)et une pour le client web. À chaque classe liée à un écran est associé un fichier .xml qui définit l'aspect visuel de l'écran présenté et appelé activity\_xxxx. Ce fichier est chargé via les méthodes "onCreate()" des différentes classes d'activité.

	Les applications Android se basent sur un système d'"activités" et une activité correspond à une fenêtre visible par l'utilisateur. L'application en possède trois : La fenêtre d'accueil (MainActivity), la fenêtre d'affichage des données de localisation du drone (TrackActivity) et celle des paramètres (ParamsActivity).
	
	\begin{minipage}{0.35\linewidth}
 		\centering
  		\includegraphics[width=0.6\textwidth]{Ecran_accueil}
  		\captionof{figure}{Écran d'accueil}
	\end{minipage}
	\begin{minipage}{0.45\linewidth}

	La fenêtre ou écran d'accueil ne comprend qu'un bouton permettant de lancer l'activité de localisation et donc le client web et un bouton pour quitter l'application, accompagnés du logo du projet. Un menu déroulant a été implémenté de façon à pouvoir accéder aux paramètres.
	\end{minipage}
	~\\	
	
	\begin{minipage}{0.35\linewidth}
 		\centering
 	 \includegraphics[width=0.6\textwidth]{Ecran_loc}
 		 \captionof{figure}{Écran de localisation}
	\end{minipage}
	\begin{minipage}{0.45\linewidth}
	 La fenêtre ou écran de localisation est un peu plus complexe. Pour l'utilisateur, les trois éléments clef de l'interface sont : les informations concernant l'état du système, le bouton "rafraîchir" et le bouton "retour". Un indicateur visuel est aussi présent en cas de détection d'intrusion.
	\end{minipage}
	~\\
	
	\begin{minipage}{0.35\linewidth}
  	\centering
 	 \includegraphics[width=0.6\textwidth]{Ecran_set}
 		 \captionof{figure}{Écran de localisation}
	\end{minipage}
	\begin{minipage}{0.45\linewidth}
	La fenêtre de paramètres permet de modifier les paramètres de connexion du client comme l'adresse et le port du serveur visé.
	\end{minipage}
	~\\	
	
\subsection{Détails concernant la fenêtre de localisation}

	La fenêtre de localisation comporte une particularité essentielle. En effet, en même temps que celle ci s'affiche, le client web (classe Client) est lancé en arrière plan grâce à une "AsynchTask" (Tâche asynchrone). L'intérêt d'un tel mode de fonctionnement présente l'avantage de ne pas avoir à créer de "Thread" (Fil d'exécution) supplémentaire et de découpler le serveur de l'affichage. Ainsi, en cas d'erreur de connexion ou de problème irrécupérable, l'interface continue de répondre et permet à l'utilisateur de relancer la connexion. Le système Android gèrera en parallèle l'affichage et le serveur, le fonctionnement des AsyncTask permet de s'assurer que tant que la fenêtre de l'application est visible par l'utilisateur le client fonctionnera en arrière plan.
	
	 

	

%%% Local Variables: 
%%% mode: latex
%%% TeX-master: "../rapport"
%%% End: 

\part{Organisation d'équipe}



%%%% CONCLUSION %%%%%%%%%

\chapter*{Conclusion}
\addcontentsline{toc}{chapter}{Conclusion}
\newpage

%%%% ANNEXE %%%%%%%%%%%%

\part*{Annexe}
\appendix
\nocite{*}

\chapter{Drone}

\section{Définition}
La définition suivante est extraite de futura science \cite{futura}.~\\

Un drone est un aéronef sans passager ni pilote qui peut voler de façon autonome ou être contrôlé à distance depuis le sol. Le mot « drone » est employé pour désigner des véhicules aériens, terrestres, de surface ou sous-marins, alors que la classification anglo-saxonne distingue chaque type d’appareil.  ~\\

La taille d’un drone aérien peut aller de quelques centimètres pour les modèles miniatures à plusieurs mètres pour les drones spécialisés (surveillance, renseignement, combat, transport, loisirs). L’autonomie en vol va de quelques minutes à plus de 40 heures pour les drones de longue endurance.  

\section{Type des drones}

On distingue deux types des drones selon leur utilisation et leur taille : militaires et civiles. 

\subsection{Les drones militaires}

Le concept du drone a émergé durant la Premier Guerre mondiale. À l’origine, le drone était un avion-cible à vocation militaire. Son développement a suivi le rythme des grands conflits du XXe siècle : Seconde Guerre mondiale, guerre de Corée, du Vietnam, guerre froide, conflits au Moyen-Orient, guerre d’Irak, d’Afghanistan ou encore en ex-Yougoslavie. Les drones sont plus économiques tout en évitant de mettre en jeu la vie des pilotes et de déployer des troupes terrestres notamment pour les missions de reconnaissance, de surveillance et les attaques ciblées. Leur utilisation au sein des armées et forces de police est devenue prépondérante.  ~\\

% \subsubsection{Types des missions}

 
%En effet, 
Les missions qui leur sont dévolues sont très variées:  ~\\

\begin{itemize}
\item[1.] Écoute des signaux électromagnétiques.
\item[2.]    Observation et surveillance. 
\item[3.]    Détection de missile balistique grâce à une alerte avancée. 
\item[4.]    Relais de communication. 
\item[5.]    Illumination de cibles. 
\item[6.]    Brouillage. 
\item[7.]     Et pour certains, bombardement. 
%\item[8.] Transport de marchandise.
\end{itemize}
 

\subsection{Les drones civils }

Dans le civil, de nombreux domaines (cinéma, télévision, agriculture, environnement, etc.) ont vu les drones susciter des applications inédites grâce à leur capacité à embarquer des appareils photo, des caméras, des caméras infrarouge ou des capteurs environnementaux. Plusieurs sociétés spécialisées dans le transport (DHL, UPS, Allship, La Poste) ainsi que le géant du e-commerce Amazon travaillent sur des concepts de drones-livreurs. Ce type de service a été introduit en 2015 aux Émirats arabes unis pour la livraison de documents officiels.  ~\\

Les drones de loisir ont connu un essor important à partir des années 2010 avec l’arrivée d’appareils miniaturisés, abordables et suffisamment maniables pour être accessibles aux novices. En France, l'utilisation des drones est réglementée par le Code de l’aviation civile, le Code des transports et deux arrêtés émis en 2012. 

%[1]: http://www.futura-sciences.com/magazines/espace/infos/dico/d/aeronautique-drone-6174/ 


    

\section{Fréquence}

Il y a 3 fréquences utilisées pour la communication entre le drone et le télécommande: ~\\
\begin{itemize}
\item 1,3 GHz
\item    2,4 GHz 
\item    5,8 GHz 
\end{itemize}  
~\\


Pour une meilleure qualité d’image, mieux vaut utiliser la bande de 5,8 GHz sachant que le 2,4 GHz permet de plus longues distances mais une qualité d’images plus faible. 

Afin d’éviter les interférences, mieux vaut éviter d’émettre sur la même bande que votre radiocommande. 

Le 2,4 GHz est la même bande que le Wifi, donc à éviter en milieu urbain et en cas d’utilisation évitez d’allumer votre smartphone à côté. 

\section{Types de modulations }

Mais il est impossible d’envoyer un signal numérique tel quel par ondes électromagnétiques. Le signal a besoin d’être modulé, c'est-à-dire transformé d’un signal numérique à un signal analogique. L’opération inverse, la démodulation, se fait par la suite après la réception du signal par la station afin d’obtenir une information exploitable. 

 
\begin{figure}[h]
  \centering
  \includegraphics[width=0.7\textwidth]{modulation}
  \caption{types de modulations}
\end{figure}
 


Il existe plusieurs types de modulation dont voici les plus connus: ~\\

 
\textit{AM (Amplitude Modulation) :} Pour faire passer de l’information, on modifie l’amplitude du signal au cours du temps. ~\\

 

\textit{FM (Frequency Modulation) :} Pour faire passer de l’information on modifie la fréquence du signal au cours du temps. ~\\

 

\textit{PM (Phase Modulation) :} C’est la technologie la plus utilisés pour les transmissions radio. Ici on fait passer l’information en modifiant la phase du signal. L’amplitude et la fréquence du signal restent donc fixes. Il existe plusieurs types de modulations par phases : BPSK \footnote{Binary Phase Shift keying}, QPSK\footnote{Quadratic PSK} … BPSK est binaire, on peut donc utiliser deux phases différentes et ainsi faire passer deux types d’informations (0 ou 1). QPSK est quadratique c'est-à-dire que l’on peut utiliser quatre phases différentes donc que l’on peut transmettre quatre types d’informations (00, 01, 10,11). Plus le nombre de phases augmente plus le nombre d’informations pouvant être véhiculé augmente. Mais cela rend le signal plus  exposé aux erreurs car les phases sont de plus en plus proches. ~\\

 

Lorsque l’on modifie la phase d’un signal, on « décale » le signal dans le temps. Graphiquement, cela se traduit par une translation de la courbe du signal sur l’axe des abscisses. La modulation de phase s’exprime en degrés ou en radians. Ainsi 360\up{0} (ou 2n radians) correspond à un décalage d’une période. On dit de deux signaux qu’ils sont en phase lorsqu’ils se superposent. 

 

\begin{figure}[h]
  \centering
  \includegraphics[width=0.7\textwidth]{signal}
  \caption{Exemple d'un signal modulé en phase avec un changement de phase de 180\up{0}}
\end{figure}

 

 

Une fois que ce signal modulé arrive à destination, on opère l’action inverse : la démodulation. Cela permet d’avoir une information numérique exploitable et d’ensuite pouvoir afficher l’image sur un écran afin que le pilote puisse voir ce que drone \og voit \fg{}. 

 

La transmission de données par ondes électromagnétiques ne se fait pas sans erreur, surtout lorsque celles-ci traversent plusieurs milliers de kilomètres dont l'atmosphère terrestre. Ces erreurs surviennent cependant en groupe et de façon très localisé. Mais de simples algorithmes de correction d’erreur comme le FEC\footnote{Forward Error Correction} peuvent assurer de bonnes conditions de transmissions. 

\section{Législations}



La législation en France impose une puissance d’émission maximale de 25 mW dans la fréquence des 5.8 GHz et 100 mW dans la fréquence 2.4 GHz. ~\\

Pour voler en immersion, il faut être 2 avec 2 radiocommandes, un \og esclave\fg{} et un \og maître\fg{} pouvant reprendre le contrôle à tout moment si besoin. ~\\

Les agents de contrôle peuvent, à tout moment, effectuer des contrôles, tant au niveau des utilisateurs amateurs que professionnels, sur le respect des caractéristiques techniques radios des drones. Une utilisation du spectre est trop dangereuse. En cas de non-respect de celles-ci, le matériel peut être saisi et un procès-verbal dressé.~\\ 

Cependant, un arrêté sur les conditions d’utilisation et des personnes capables de  piloter ces drones a été rédigé en France par les services de l’aviation civile le 10 Décembre 2009. Dans l’article 3, les drones doivent être connus des services de l’aviation civile lorsque les drones ne sont pas pour des activités sportives ou récréatives. Ils doivent être connus lorsqu’ils font partie d’une association d’aéromodélisme mais aussi lorsqu’ils évoluent à plus de 150 mètres d’altitudes (ils doivent fournir des justificatifs prouvant le besoin et disposer de précautions particulières, et surtout les vols de nuit ne sont pas autorisés. ~\\

L’utilisation de drones civils est très contraignante. En effet, l’utilisateur doit de demander une autorisation au service de l’aviation civile, pour chaque vol, plusieurs semaines à l’avance ce qui est presque impossible, car les drones dépendent de la météo. De plus, aucune autorisation n’est délivrée à l’heure actuelle pour des survols avec des drones civils dans des zones civiles de zones habitées ou avec un rassemblement de personnes (manifestation). Or, c’est dans ces zones que sont prises la majeure partie des photographies. Il est faux de croire qu’en dessous de 150 mètres d’altitudes il n’y a pas de règlementation et que les avions et hélicoptères avec pilote ne peuvent pas descendre à cette altitude. A cette altitude, il est nécessaire de posséder une autorisation, car sur le territoire français des milliers d’heures de vols se font entre 50 et 150 mètres d’altitudes avec des avions et hélicoptères traditionnels (vols basse altitude d’entrainements militaires, interventions d’hélicoptères de secours).  Les aéronefs traditionnels  ne peuvent pas éviter les drones à cette altitude à cause de leur faible taille, presque invisible en vol. 



%%% Local Variables: 
%%% mode: latex
%%% TeX-master: "rapport_analyse"
%%% End: 

\input{Partie/S3/Partie_Organisation}
\chapter{Correspondance}
\label{chap:mail}


\includepdf[pages={1,2}]{./images/mail.pdf}


%%% Local Variables: 
%%% mode: latex
%%% TeX-master: "../../rapport"
%%% End: 




\chapter{Documentation Technique du Raspberry Pi B+}
\label{annexe:rpi}

\section{Définition}


\textit{\og Le Raspberry Pi est un nano-ordinateur monocarte à processeur ARM conçu par le créateur de jeux vidéo David Braben, dans le cadre de sa fondation Raspberry Pi.}

\textit{Cet ordinateur, qui a la taille d'une carte de crédit, est destiné à encourager l'apprentissage de la programmation informatique2 ; il permet l'exécution de plusieurs variantes du système d'exploitation libre GNU/Linux et des logiciels compatibles. Il est fourni nu (carte mère seule, sans boîtier, alimentation, clavier, souris ni écran) dans l'objectif de diminuer les coûts et de permettre l'utilisation de matériel de récupération.}

\textit{Son prix de vente était estimé à 25 \$, soit 19,09 \euro, début mai 2011. Les premiers exemplaires ont été mis en vente le 29 février 2012 pour environ 25 \euro. Début 2015, plus de cinq millions de Raspberry Pi ont été vendus. De multiples versions ont été développées (voir la liste ci-dessous), on trouve les dernières à un peu plus de 25 \euro pour le B+, à un peu plus de 30 \euro pour le Pi 2 (2015) et à un peu plus de 45 \euro pour le Pi 3 (2016)\fg{}} Wikipédia \cite{wiki_rpi}

\section{Caractéristique et connectiques}
\begin{tabular}[c]{|l|l|}
\hline
\multicolumn{2}{|c|}{Caractéristiques}\\
\hline
Micro-contrôleur &	Broadcom BCM2835 ARM1176JZFS\\
Vitesse d'horloge& 	700 MHz\\
RAM &	512 Mo\\
\hline
\multicolumn{2}{c}{}\\
\hline
\multicolumn{2}{|c|}{Connectiques}\\
\hline
Port(s) USB &	4\\
Port Ethernet / RJ45 &	1\\
Connecteur(s) audio analogique &	1 sortie jack 3,5 mm\\
HDMI 	&1\\
Port pour carte mémoire 	&1 port micro SD\\
Alimentation &	via port micro USB 5V\\
\hline
\end{tabular}

\newpage
\section{Schéma technique}



\begin{figure}[!h]
  \centering
  \includegraphics[width=\textwidth]{raspberrypi_doc_mecanique}
  \caption{Dessin mecanique}
\end{figure}

\begin{figure}[!h]
  \centering
  \includegraphics[width=\textwidth]{raspberrypi_doc_schema}
  \caption{Schéma technique}
\end{figure}

%%% Local Variables: 
%%% mode: latex
%%% TeX-master: "../rapport"
%%% End: 


\chapter{Documentation Technique à Raspbian}
\label{annexe:raspbian}

Raspbian (recommended for Raspberry Pi 1) – is maintained independently of the Foundation; based on the Debian ARM hard-float (armhf) architecture port originally designed for ARMv7 and later processors (with Jazelle RCT/ThumbEE and VFPv3), compiled for the more limited ARMv6 instruction set of the Raspberry Pi 1. A minimum size of 4 GB SD card is required for the Raspbian images provided by the Raspberry Pi Foundation. There is a Pi Store for exchanging programs.

    The Raspbian Server Edition is a stripped version with fewer software packages bundled as compared to the usual desktop computer oriented Raspbian.

    The Wayland display server protocol enables efficient use of the GPU for hardware accelerated GUI drawing functions.[104] On 16 April 2014, a GUI shell for Weston called Maynard was released.

    PiBang Linux – is derived from Raspbian.

    Raspbian for Robots – is a fork of Raspbian for robotics projects with Lego, Grove, and Arduino.

%%% Local Variables: 
%%% mode: latex
%%% TeX-master: "../rapport"
%%% End: 

\newpage
 \listoffigures
 \printindex
 \bibliographystyle{frplain}
  \bibliography{biblio}

\end{document}
%%%%%%%%%%%%%%%%% FIN DU DOCUMENT
