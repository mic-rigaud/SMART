\chapter{Ingénierie Système}

\section{Analyse Fonctionelle}

\subsection{Diagramme Pieuvre}

Le diagramme pieuvre permet de mettre en évidence rapidement la fonction principale du système et les principales contraintes qui s'appliquent sur le système. Ce dernier est représenté par l'ovale central et l'ensemble des éléments extérieurs ayant une influence sont matérialisés tout autour. Les différentes relations sont appelées les fonctions de contraintes qui naissent d'une contrainte imposée par un élément extérieur « météo », de l'existence d'un produit déjà existant « un autre drone émettant des ondes » ou encore d'une exigence particulière de l'utilisateur voire de la présence de normes et de législations, de limitations lié au budget ou du type d'alimentation énergétique nécessaire.


\includegraphics[width=0.90\textwidth]{Diagramme_pieuvre_corr.png}
\captionof{figure}{Diagramme pieuvre}


\subsection{Diagramme FAST}

Ce diagramme  présente la  manière de penser et d'agir. Le diagramme FAST se construit de gauche à droite, dans une logique du pourquoi au comment. On développe les fonctions de service du système en fonctions techniques. On choisit des solutions pour construire finalement notre système. On a mentionné les fonctions techniques chacune  à part pour trouver la solution convenable qui nous permet à la fin la réalisation finale du système. En utilisant des outils et méthodes déjà existant, on a trouvé des solutions qui satisfont les fonctions demandés.
L’antenne goniomètre était l’une des solutions les moins chères pour la détection du drone, à condition d’avoir au minimum deux antennes pour préciser la position et la vitesse du drone. Dès la détection du drone, il sera alors possible de déterminer sa position, d'enregistrer cette position via un logiciel dédié (MATLAB)

~\\

\includegraphics[width=1\textwidth]{FAST.png}
\captionof{figure}{Diagramme pieuvre}
\parindent=15pt

\chapter{Tests unitaires}

\section{Raspberry Pi}
La documentation technique lié a notre Raspberry Pi es situé en annexe à la page \pageref{annexe:rpi}
~\\

\includegraphics[width=\textwidth]{Test_unitaire/Rpi/img5.jpg}
\captionof{figure}{Notre Raspberry Pi B+}

~\\
\parindent=15pt

Pour s'assurer que notre Raspberry Pi répond aux spécifications fonctionnelles et qu'il fonctionne correctement en toutes circonstances pour notre projet, nous y avons réalisé des tests unitaires.

Après avoir enfin installé le système d'exploitation Raspbian\footnote{La documentation lié à Raspbian est situé en annexe à la page \pageref{annexe:raspbian}} sur notre Raspberry Pi B+, nous avons tenté de tester les ports GPIO. Pour cela, dans un premier temps, nous avons allumé des LED grâce à un script python à travers différents ports GPIO. Sur la figure \ref{figure:led}, on peut observer que nous avons allumé une LED grâce au port 22.
~\\

Dans notre projet le Raspberry Pi sera placé entre le radio-goniomètre à effet Doppler et l'utilisateur. Il aura deux taches, corréler les données entre tous les dispositifs pour obtenir la position du drone et afficher le résultat à l'utilisateur. Pour cela il doit récupérer la direction qui est donné par le Montréal 3v2. Cette position est donnée à travers des LED (voir figure \ref{figure:ledMontreal}). Nous allons donc placer le Raspberry Pi au niveau des LED pour obtenir les informations délivré par le Montréal 3v2. % TODO : A FINIR

\begin{figure}[!h]
  \includegraphics[width=\textwidth]{Test_unitaire/Rpi/img3.jpg}
  \caption{Allumage d'une LED par Raspberry Pi}
  \label{figure:led}
\end{figure}

\begin{figure}[!h]
  \centering
  \includegraphics[width=0.8\textwidth]{Test_unitaire/Rpi/led.png}
  \caption{Méthode de connexion des leds dans le Montréal}  
  \label{figure:ledMontreal}
\end{figure}
\begin{figure}[!h]
  \centering
  \includegraphics[width=\textwidth]{Test_unitaire/Rpi/img4.jpg}  
  \caption{Système modélisant une LED}
  \label{figure:test}
\end{figure}

La sortie du Montréal 3v2 est décrite à la figure \ref{figure:ledMontreal}. On constate que le pic qui permet l'affichage à 12 sortie qui lui permet de gérer 24 LED. Pour connaître quelle LED est allumé, il faut savoir laquelle des entré A1,A0,B7,B6,B5,B4 à un front montant et laquelle des entrées B0,B1,B2,B3,A3,A2 à un front descendant.

Pour modéliser une LED en entré du Raspberry Pi, nous avons positionné 2 boutons poussoirs (voir figure \ref{figure:test}). Le premier permets de réaliser le front montant et le second le front descendant. Ainsi en positionnant ces boutons au bon endroit par rapport au port GPIO du raspberry il est possible de connaître quelle LED on a simuler.

Nous avons réaliser un script python qui lié les entrées du raspberry avec les sortie du pic. Puis nous avons tester en simulant une LED comme décrit précédemment.

On peut constater que l'expérience est un succès car le raspberry pi nous renvoie bien le numéro de la LED que nous voulions tester.




%%%%%%%%%%%%%%%%%%%%%%%%%%%%%%%%%%%%%%%%%%%%%%%%%%%

\section{PIC}
\label{sec:pic}

Dans le schéma du Montréal 3v2 nous avons pu constater qu'il y avait 3 PIC programmés. Nous avons commandé les PIC programmés au près de l'entreprise F1LVT \cite{montreal}.


\begin{figure}[!h]
  \centering
  \includegraphics[width=\textwidth]{Test_unitaire/Pic/pic}
  \caption{3 PIC programmés}
  \label{fig:pic}
\end{figure}

Nous avons ensuite imaginé et réalisé des tests unitaires sur chacun des PIC pour vérifier qu'ils ont bien été programmés et qu'aucune erreur n'est apparu sur ce système de décision critique pour le système.

\subsection{PIC16F628A}
\label{sec:picled}

Ce PIC sert à réaliser l'affichage sur les LED. Pour tester ce PIC, nous avons réaliser le montage de la figure \ref{fig:picled}. On peut voir à la figure \ref{fig:schemapic} le schéma de montage du PIC sur le Montréal 3v2.

\begin{figure}[!h]
  \centering
  \includegraphics[width=\textwidth]{Test_unitaire/Pic/picled}
  \caption{Schéma electrique du test unitaire}
  \label{fig:picled}
\end{figure}


Pour tester ce composant, nous avons donc choisi de monter une partie des LED situés en sorti, de configurer le \og clock\fg{} sur un signal carré de fréquence 1 MHz, et de faire varier la fréquence de l'entrée \og data\fg{}.

Malheureusement nous n'avons pas pu observer de LED s'allumer pendant notre expérience.

\begin{figure}[!h]
  \centering
  \includegraphics[width=\textwidth]{Test_unitaire/Pic/schemapic}
  \caption{Schéma de montage du pic sur le Montréal 3v2}
  \label{fig:schemapic}
\end{figure}

\subsection{PIC12F675}
\label{sec:picclock}

\subsection{PIC18F4520}
\label{sec:piccontrol}






%%% Local Variables: 
%%% mode: latex
%%% TeX-master: "../rapport"
%%% End: 