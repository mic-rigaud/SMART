
\chapter{Interface Web}
\label{Logiciel}

Une foi le capteur terminé nous nous sommes penchés sur l'interface Homme/Machine. Dans une volonté de délivrer à l'utilisateur une interface agréable et lisible, nous avons décidé de proposer dans un premier temps une interface web.

\section{Analyse}
\label{sec:uml}

Dans un premier temps nous nous sommes penché sur une  phase d'analyse.

Dans la phase d’analyse, on cherche d’abord à bien comprendre et à décrire de façon précise les besoins des utilisateurs ou des clients concernant cette interface. Que souhaitent-ils faire avec le logiciel ? Quelles fonctionnalités veulent-ils ? Pour quel usage ? Comment l’action devrait-elle fonctionner ? C’est ce qu’on appelle \og l’analyse des besoins\fg{}. Après validation de notre compréhension du besoin, nous imaginons la solution. C’est la partie analyse de la solution.

Dans la phase de conception, on apporte plus de détails à la solution et on cherche à clarifier des aspects techniques, tels que l’installation des différentes parties logicielles à installer sur du matériel. Pour réaliser ces deux phases dans un projet informatique, nous utilisons des méthodes, des conventions et des notations. UML fait partie des notations les plus utilisées aujourd’hui. Pour faciliter à nos clients d’obtenir la direction des drones on a créé une interface web qui répond à leur besoin.

\subsection{Uml}

Pour décrire au mieux ce besoin, nous avons commencé par réaliser un cas d'utilisation de l'interface (figure \ref{fig:use_case}).
\begin{figure}[!h]
  \centering
  \includegraphics[width=\textwidth]{use_case}
  \caption{Cas d'utilisation de l'interface}
  \label{fig:use_case}
\end{figure}

\newpage
Ensuite, nous avons cherché a réaliser un diagramme de classe de notre interface. Pour cela nous avons défini 3 classes principales:

\begin{itemize}
\item index.php, qui réalise l'affichage dans un navigateur
\item serveur.py, qui récupère les données de chacun des radiogoniomètres
\item client.py, installé sur chaque radiogoniomètres il envoie les données des capteurs à travers un socket au serveur.
\end{itemize}

Le diagrammes de classe de la figure \ref{fig:class}, montre ce fonctionnement.

\begin{figure}[!h]
  \centering
  \includegraphics[width=\textwidth]{class_diagram}
  \caption{Diagramme de classe}
  \label{fig:class}
\end{figure}



\section{Conception}

\subsection{Client-Serveur}

Pour commencer nous avons cherché à donner une interface à nos \rpi pour communiquer, c'est l'interface Client/Serveur.
~\\

Cette interface a été codé en python. Nous avons choisi le python car c'est un langage souple et rapide. Nous avons réalisé la connection à l'aide d'un socket TCP/IP. De plus, le serveur se base sur du multi thread pour accepter plusieurs client. Le client se connecte donne son identifiant puis sa communication est placé dans un thread. Ainsi, on a un serveur qui peut accepter une infinité de client.
~\\


\begin{minipage}[h]{0.45\linewidth}

\begin{lstlisting}
./serveur.py
Serveur pret, en attente de requetes ...
Client RP1 connecte, adresse IP 127.0.0.1, port 51632.
RP1> led21
donc la direction 200
[...]
Client RP1 deconnecte.
[...]
Fin du serveur
\end{lstlisting}  
\end{minipage}\hfill
\begin{minipage}[h]{0.45\linewidth}
  
\begin{lstlisting}
./client.py
Connexion etablie avec le serveur.
Vous etes connecte. Envoyez vos messages.
la led allume: led21
[...]
Connexion interrompue.
\end{lstlisting}

\end{minipage}





\subsection{Web}


\subsection{Treillis de détection}

Pour répondre au besoin de détection et s’assurer d’un correct positionnement de la cible, la mise en
place d’une couverture de détection répondant à nos besoins était nécessaire. Les critères retenus
pour cette dernière sont les suivants :

\begin{itemize}
\item A l’intérieur de la zone de détection, la cible doit être en permanence sous la couverture de détection de 4 radiogoniomètres
\item Tenter une optimisation de la couverture afin d’éviter l’installation d’un trop grand
nombre de radiogoniomètre
\end{itemize}

Très peu de sujets similaires ont pu être trouvé bien que le problème soit récurrent dans de
nombreux projets.

Cependant, après plusieurs essais, le choix de treillis fut le suivant :

\begin{minipage}{0.45\linewidth}

  \centering
  \includegraphics[width=\textwidth]{treillis_explication}
  \caption{Cercle de détection}
  ~\\
\end{minipage}
\begin{minipage}{0.45\linewidth}
  \begin{itemize}
  \item Dans les deux cas, les distances d et d’ représente la distance de couverture maximale d’une antenne pour une configuration de treillis particulière, distance au-delà de laquelle nous ne sommes pas sûr d’assurer la détection d’un drone.
  \item Chaque croix noire représente une antenne. Ces dernières formes ainsi la zone de détection, zone à l’intérieure de laquelle, le drone se doit d’être repéré.
  \end{itemize}
\end{minipage}


Initialement, nous avions prévu que les antennes radiogoniométrique assureraient la détection
jusqu’à son plus proche voisin. Cependant, cette configuration ne permet pas d’assurer qu’un drone
traversant la zone soit sous la couverture d’au moins quatre antennes en tout temps. Nous avons
donc choisi la configuration représentée par le cercle rouge, c’est-à-dire en rapprochant les antennes
les unes des autres. On obtient ainsi la couverture suivante :
~\\

\begin{minipage}{0.45\linewidth}
  \centering
  \includegraphics[width=\textwidth]{cercle}
  \caption{Maillage de détection}
\end{minipage}
\begin{minipage}{0.45\linewidth}
  Dans cette configuration, les zones de plus faible couverture
sont situées sur les antennes elles même. En effet, au-dessus
de chaque antenne, la couverture n’est assurée que par
quatre d’entre elles. En dehors de celle-ci, la couverture est
assurée par cinq à six antennes.
\end{minipage}


\subsection{Triangulation}

Les différents radiogoniomètres nous donnant un gisement de la détection, nous pouvons donc
réaliser une triangulation de la position du drone lorsqu’un nombre suffisant d’antenne détecte le
drone. Bien qu’une première estimation de la position peut-être obtenu à partir de deux drones,
nous considérons que le drone doit être détecté par au moins quatre senseurs pour que la position
soit acceptable.
~\\

Cependant, avant de pouvoir réaliser toute triangulation, l’acquisition des points d’intersection entre
les droites de détection issue du gisement fourni par les différents radiogoniomètres est nécessaire.
Pour se faire, à partir des angles, nous formulons une équation de droite plan, passant par les
radiogoniomètres respectifs, et tentons de trouver une solution à chaque système composé de deux
droites.
~\\

\begin{wrapfigure}{r}{0.5\textwidth}

  \includegraphics[width=0.5\textwidth]{triangulation1}
  \caption{triangulation}
\end{wrapfigure}

  Suite à la résolution de ces différents systèmes, nous
obtenons une liste de différentes solutions, solutions ici
schématisé avec des points de couleur grise. Nous observons
que, du fait de la portée de détection, ainsi que la géométrie
de notre treillis, certains points sont incohérents ou très
imprécis.

~\\
Notre première approche fut donc de positionner le drone à
la moyenne de l’ensemble des positions solutions d’un des
systèmes précédent. Cependant, après simulation, il s’est
avéré que, de par l’imprécision relative des
radiogoniomètres (les angles sont donnée à 5\up{o} près), la moyenne de donnait qu’une idée toute relative de la position du drone et souvent loin de la vérité, et
cela à cause d’intersections multiples entre les droites de détection.
Pour corriger cela nous avons fait le choix d’utiliser la médiane afin de supprimer tous les résultats
incohérents. A partir de la médiane des résultats, nous appliquons un gabarit circulaire et retenons
tous les points d’intersection compris dans ce gabarit. Une moyenne est alors appliquée à l’ensemble
de ces résultats nous permettant d’obtenir un résultat plus cohérent et moins sensible aux erreurs.



\begin{figure}[!h]
  \centering
  \includegraphics[width=\textwidth]{interface}
  \caption{Interface Web}
  \label{fig:interface}
\end{figure}


%%% Local Variables: 
%%% mode: latex
%%% TeX-master: "../rapport"
%%% End: 
