
\chapter{Documentation Technique à Raspbian}
\label{annexe:raspbian}

\section{Définition}


\textit{\og Raspbian est un système d’exploitation libre basé sur la distribution GNU/Linux Debian, et optimisé pour le plus petit ordinateur du monde, la Raspberry Pi.}

\textit{Raspbian ne fournit pas simplement un système d’exploitation basique, il est aussi livré avec plus de 35 000 paquets, c’est-à-dire des logiciels pré-compilés livrés dans un format optimisé, pour une installation facile sur votre Raspberry Pi via les gestionnaires de paquets.
La première version des 35 000 paquets Raspbian, optimisés pour la Raspberry Pi, a été achevée en Juin 2012.}

\textit{Néanmoins, Raspbian est toujours en développement, avec une priorité à l’amélioration de la stabilité et des performances d’un maximum de paquets Debian possibles\fg{}} raspbian-france\cite{raspbian}

\section{Logo}

\begin{figure}[h]
  \centering
  \includegraphics[width=0.4\textwidth]{raspbian}
  \caption{logo raspbian}
  \label{fig:rasp}
\end{figure}


% Raspbian (recommended for Raspberry Pi 1) – is maintained independently of the Foundation; based on the Debian ARM hard-float (armhf) architecture port originally designed for ARMv7 and later processors (with Jazelle RCT/ThumbEE and VFPv3), compiled for the more limited ARMv6 instruction set of the Raspberry Pi 1. A minimum size of 4 GB SD card is required for the Raspbian images provided by the Raspberry Pi Foundation. There is a Pi Store for exchanging programs.

%     The Raspbian Server Edition is a stripped version with fewer software packages bundled as compared to the usual desktop computer oriented Raspbian.

%     The Wayland display server protocol enables efficient use of the GPU for hardware accelerated GUI drawing functions.[104] On 16 April 2014, a GUI shell for Weston called Maynard was released.

%     PiBang Linux – is derived from Raspbian.

%     Raspbian for Robots – is a fork of Raspbian for robotics
%    projects with Lego, Grove, and Arduino.

%%% Local Variables: 
%%% mode: latex
%%% TeX-master: "../rapport"
%%% End: 
