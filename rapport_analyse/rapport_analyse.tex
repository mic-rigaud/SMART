
\documentclass[a4paper, 11pt, oneside]{memoir}

%%%%% Packages %%%%%
\usepackage{lmodern}
\usepackage{palatino}
\usepackage[T1]{fontenc}
\usepackage[utf8]{inputenc}
\usepackage[french]{babel}


%%%%%%%%%%%%%%%%%%%% PACKAGE SECONDAIRE

% \usepackage{amstext,amsmath,amssymb,amsfonts} % package math
% \usepackage{multirow,colortbl}	% to use multirow and ?
% \usepackage{xspace,varioref}
\usepackage[linktoc=all, hidelinks]{hyperref}			% permet d'utiliser les liens hyper textes
\usepackage{float}				% permet d ajouter d autre fonction au floatant
% \usepackage{wrapfig}			% permet d avoir des image avec texte coulant a cote
% \usepackage{fancyhdr}			% permet d inserer des choses en haut et en bas de chaque page
\usepackage{microtype}			% permet d ameliorer l apparence du texte
%\usepackage[explicit]{titlesec}	% permet de modifier les titres
\usepackage{graphicx}			% permet d utiliser les graphiques
\graphicspath{{./images/}}		% to say where are image
% \usepackage{eso-pic} 			% to put figure in the background
\usepackage[svgnames]{xcolor}	% permet d avoir plus de 300 couleur predefini
% \usepackage{array}				% permet d ajouter des option dans les tableaux
% \usepackage{listings}			% permet d ajouter des ligne de code
% \usepackage{tikz}				% to draw figure
% \usepackage{appendix}			% permet de faire les index
% \usepackage{makeidx}			% permet de creer les index
% \usepackage{fancyvrb}			% to use Verbatim
% \usepackage{framed}				% permet de faire des environnement cadre
% \usepackage{fancybox}			% permet de realiser les cadres
%\usepackage{titletoc}			% permet de modifier les titres
\usepackage{caption}
\usepackage[a4paper, top=2cm, bottom=2cm]{geometry}
\usepackage[final]{pdfpages} 
\usepackage{eurosym}

\usepackage{graphicx}
\RequirePackage{pageGardeEnsta}	% permet d avoir la page de garde ensta

% \setcounter{secnumdepth}{2}		% permet d'augmenter la numerotation
% \setcounter{tocdepth}{2}		% permet d'augmenter la numerotation

%%%%%%%%%%%%%%%%%% DEFINITION DES BOITES
\newcounter{rem}[chapter]

\newcommand{\remarque}[1]{\stepcounter{rem}\noindent\fcolorbox{OliveDrab}{white}{\parbox{\textwidth}{\textcolor{OliveDrab}{
        \textbf{Remarque~\thechapter.\therem~:}}\\#1}}}

\newcounter{th}[chapter]

\newcommand{\theoreme}[2]{\noindent\fcolorbox{FireBrick}{white}{\stepcounter{th}
    \parbox{\textwidth}{\textbf{\textcolor{FireBrick}{Théorème~\thechapter.\theth~:}}{\hfill \textit{#1}}\\#2}}}

\newcommand{\attention}[1]{\noindent\fcolorbox{white}{white}{\parbox{\textwidth}{\textcolor{FireBrick}{
        \textbf{Attention !}}\\\textit{#1}\\}}}
%%%%%%%%%%%%%%%%%%%%%%%%%%%%%%%%%%%%%%%%%%%%%%%%%%%%%%%%%%%%%%%%%%%%%%%%% 


%% INDEX %%%%%%%%%%%%%%%%%%%%%%%%%%%%%%%%%%%%%%%%%%%%%%%%%%%%
\makeindex

%%%%% Useful macros %%%%%
\newcommand{\latinloc}[1]{\ifx\undefined\lncs\relax\emph{#1}\else\textrm{#1}\fi\xspace}
\newcommand{\etc}{\latinloc{etc}}
\newcommand{\eg}{\latinloc{e.g.}}
\newcommand{\ie}{\latinloc{i.e.}}
\newcommand{\cad}{c'est-à-dire }
\newcommand{\st}{\ensuremath{\text{\xspace s.t.\xspace}}}

%%%% Definition des couleur %%%%

\newcommand\couleurb[1]{\textcolor{SteelBlue}{#1}}
\newcommand\couleurr[1]{\textcolor{DarkRed}{#1}}


%% number page style style %%%%%%%%%%%%%%%%%%%%%%%%%%%%%%%%%%%%%%%%%%%%%%%%%%%%%%

\pagestyle{plain}
% \pagestyle{empty}
% \pagestyle{headings}
% \pagestyle{myheadings}



%% chapters style %%%%%%%%%%%%%%%%%%%%%%%%%%%%%%%%%%%%%%%%%%%%%%%%%%%%%%
%% You may try several styles (see more in the memoir manual).

% \chapterstyle{veelo}
% \chapterstyle{chappell}
% \chapterstyle{ell}
% \chapterstyle{ger}
% \chapterstyle{pedersen}
% \chapterstyle{verville}
\chapterstyle{madsen}
% \chapterstyle{thatcher}


%%%%% Report Title %%%%%
\title{Analyse fonctionnelle}
\author{Équipe Smart}
% \author{\textsc{Rigaud Michaël} \textsc{D'Acremont Antoine} \textsc{Cotten Guillaume} \textsc{Legay Kevin} \textsc{Aya Kenaan} \textsc{Mohamed Shehade}}
\date{\today}
\doctype{Rapport}
\promo{promo 2017}
\etablissement{\textsc{Ensta} Bretagne\\2, rue François Verny\\
  29806 \textsc{Brest} cedex\\\textsc{France}\\Tel +33 (0)2 98 34 88 00\\ \url{www.ensta-bretagne.fr}}
\logoEcole{\includegraphics[height=4.2cm]{logo_ENSTA_Bretagne_Vertical_CMJN}}



%%%%%%%%%%%%%%%%%% DEBUT DU DOCUMENT
\begin{document}

\maketitle
\thispagestyle{empty}
\newpage

\tableofcontents

\newpage
%%%%%%%%%%%%%%%%% INTRODUCTION

\chapter*{Introduction}
\addcontentsline{toc}{chapter}{Introduction}

Ce document constitue le rapport de l'analyse fonctionnelle du projet Smart.
L'équipe Smart est constituée de Rigaud Michaël, D'Acremont Antoine, Cotten Guillaume, Legay Kevin, Aya Kenaan, et Mohamed Shehade.

Le projet Smart a pour but de mettre en place un système de détection et de neutralisation de drones.

Compte tenu du temps imparti, nous avons choisi de nous concentrer sur la détection d'un drone. Pour réaliser cette détection, nous utiliserons un ensemble de goniomètre permettant de réaliser la localisation d'un drone. Ce projet étant nouveau, l'ensemble des recherches et la réalisation du système devront être mené dans le temps imparti.

\newpage

%%%%%%%%%%%%%%%%%%%%%%%% 

\part{Analyse fonctionnelle}

%\chapter{État de l'art des technologies}

\chapter{Présentation du contexte}

Dans le domaine de la détection de drones, après recherche littéraires et numérique, nous en avons conclu qu'il existait plusieurs types de détection: par acoustique, par méthodes optiques et par radiogoniométrie.
Ces méthodes possèdent chacunes leurs avantages et leurs inconvéniants que nous allons spécifier ci-dessous.


\section{Acoustique}

Plusieurs entreprises proposent des outils de détection des drones. Ces derniers se présentent sous forme de boîtiers reliés à des micros, positionnés en hauteur: c'est par le son de leurs hélices que les drones sont repérés, dans un rayon d'une centaine de mètres. Une alerte est alors envoyée sur un ordinateur ou par un SMS. Avantage: le système ne s'occupe pas des ondes, et peut détecter les drones autopilotés (voir plus bas). Problème: le bruit de fond doit être inférieur à un certain seuil, ce qui le rend difficilement utilisable en milieu urbain. De plus, pour des raisons d'échos, la multiplication de récepteurs est nécessaire afin de pouvoir filtrer le signal. Enfin, il est nécessaire de disposer préalablement d'une base de données des signatures acoustiques des différents drones qui peuvent émettre sur un domaine de fréquences acoustiques larges.

Cependant, ce système présente des failles. En effet, il est assez simple pour un drone de parer ce système de détection. Par la simple émission d'une onde sonore couvrant sa propre signature acoustique, un drone passerait totalement inaperçu.

Certains systèmes utilisent aussi une analyse fréquentielle poussée du signal afin de détecter les moteurs en fonction de leurs fréquences de fonctionnement.

Au-delà de cet aspect, il présente un avantage et des plus importants, son coût. En effet, un tel système est très économique à produire. Actuellement diverses solutions actives comme passives sont déjà présentes sur le marché. Ces solutions sont orientées vers une utilisation domestique et non professionnelle pour les raisons évoquées précédemment. Leur prix se situe aux alentours de 100 dollars pour un modèle classique, mais la multiplication des solutions tant à réduire le prix d'un tel système. 

\section{Optique}

Une caméra normale a besoin de lumière pour produire une image, une caméra thermique (ou infrarouge) peut capter de très faibles différences de température et les convertir en une excellente image thermique sur laquelle les plus petits détails sont visibles. Contrairement à d'autres technologies, comme l'amplification de lumière qui nécessite une petite quantité de lumière pour produire une image, l'imagerie thermique permet de voir dans l'obscurité totale. Elle ne nécessite aucune source de lumière.

Depuis qu'il est possible de produire une image lisible dans l'obscurité totale, la technologie de l'imagerie thermique permet de voir et de cibler les forces ennemies dans la nuit la plus noire. Les caméras thermiques voient à travers la brume, la pluie et la neige. Elles voient aussi à travers la fumée, ce qui était particulièrement intéressant pour l'armée.\cite{optique}

En mode passif, \emph{des caméras thermiques d'observation savent repérer un drone de 50 cm d'envergure à une distance d'environ 1 km, de jour comme de nuit} . Lorsqu'un drone entre dans son champ de vision, des algorithmes identifient son image. La forme, la couleur et la géométrie de l'objet permettent de distinguer le drone d'éventuels oiseaux et lancer une alerte, à condition qu'il n'y ait pas d'obstacle entre la caméra et lui.

En mode actif, on peut éclairer une scène à $360^{\circ}$ avec un laser. \emph{Les photons, les particules de lumière, se réfléchissent sur l'appareil, le signal est récupéré et analysé.} D'une portée similaire à celle de la caméra, le laser a l'avantage de \emph{décamoufler} (observation à travers brouillard, pluie ou filet de camouflage), de livrer la distance précise de l'objet, et de le reconstituer en imagerie 3D.Une fois le drone suffisamment proche, une caméra \textit{classique} avec un opérateur humain peuvent prendre le relai pour vérifier visuellement la nature de l'intrus et éventuellement passer à la phase de neutralisation.


\section{Radar}
Le radar (de l'anglais RAdio Detection And Ranging) est un système qui utilise les ondes électromagnétiques pour détecter la présence d'objets. Le radar émet des ondes, elles rebondissent sur les objets rencontrés et il est possible de mesurer leur distance, la direction, l'altitude ainsi que la vitesse en analysant le signal renvoyé. Les modèles Doppler peuvent ainsi détecter les objets en mouvement : avion, hélicoptère et certains modèles de drones, même « légers ». C'est le cas du radar Squire de Thales Air Systems. 

~\\

\includegraphics[width=\textwidth]{radar}
\captionof{figure}{Le radar portable Squire de Thales Air Systems}

Il existe néanmoins certains drones construits en carbone pouvant être perméables à certaines ondes radars et ainsi indétectable par cette technologie. Cependant le "radar passif", radar exploitant les variations d'ondes électromagnétiques en milieu urbain, telles que les ondes de la TNT, pourrait être exploité en milieu urbain.



\section{Radiogoniométrie}

Parmis les méthodes pour détecter un drone on peut citer la radiogoniométrie. Le principe de la radiogoniométrie est de mesurer la direction d'arrivée d'une onde électromagnétique polarisée incidente sur un réseau de capteur, par rapport à une direction de référence. Les radio-goniomètres sont donc des détecteurs passifs. 

La radiogoniométrie possède de nombreuses applications. Cependant, en interception, la radiogoniométrie permet de localiser un émetteur inconnu soit en employant plusieur récepteurs en des positions différentes, soit par calcul en fonction de la cinématique preopre du récepteur. 

On distingue deux types de goniomètres: les goniomètres à une dimension qui n'estiment que le gisement ou l'azimut, et les goniomètres à deux dimensions qui estiment le gisement ou azimut ainsi que l'élévation. 


Dans le cas d'une détection de drones, le radio-goniomètre réalise une écoute de l'environnement avec un balayage de fréquences. Lorsque le drone émettra avec la personne qui le guide on pourra ainsi le localiser précisément.

Seulement, la radiogoniométrie a des failles. En effet, il existe sur le marché des drones auto-pilotés qui n'émettent pas car ils chargent avant le début de leur vol leurs trajectoires. Ainsi il n'y a pas de communication avec un quelconque utilisateur, et donc il n'y a aucun signal émis. Il est donc impossible de les localiser à l'aide de cette technique.

Mais cette technique possède aussi ses avantages. C'est une technique passive et donc indécelable. C'est d'ailleurs pour cela que c'est une technique très utilisée dans la guerre électronique. 




\section{Synthèse}

Ainsi, la meilleure solution serait de réaliser un détecteur à base de ces trois modes de détection. C'est d'ailleurs pourquoi les produits les plus performants existant sur le marché utilisent un mélange de ces trois technologies. On peut notamment citer le cas du système drone-detector \cite{dronedetector}.

Néanmoins nous avons choisi pour ce projet de nous concentrer, dans un premier temps, sur une détection uniquement à base de radiogoniométrie.

% Ici il va falloir préciser plusieurs choses sur pourquoi ce choix. Notamment en précisant qu'on suppose que les drones respectent la réglementation, etc... 



%%% Local Variables: 
%%% mode: latex
%%% TeX-master: "rapport_analyse"
%%% End: 

\chapter{Analyse fonctionnelle}

\section{Interview}

Après notre interview avec notre encadrant Ali Mansour, nous avons réalisé un tableau des spécifications suivantes:

\includegraphics[width=0.8\textwidth]{interview}


\section{Tableau des spécifications}
En prenant en compte les recommandations de notre encadrant, et les recherches que nous avons réalisées, nous avons établi les contraintes et les spécifications suivantes:

%\includegraphics[width=\textwidth]{tableauSpe}
\includepdf{./images/tableauSpe.pdf}

%Compte tenu des recherches que nous avons réalisées, nous avons établi l'étude fonctionnelle suivante.

%De la synthèse de ce tableau découle le diagramme Pieuvre et les SADT suivant.

\section{Diagramme pieuvre}
~\\
~\\
~\\
~\\
~\\
~\\
\hspace{-2cm}
\includegraphics[width=1.18\textwidth]{Diagramme_pieuvre.pdf}
\captionof{figure}{Diagramme pieuvre}



\section{SADT}

\includegraphics[width=\textwidth]{SADT_A-0.pdf}
\captionof{figure}{SADT A-0}
\includegraphics[width=\textwidth]{SADT_A0.pdf}
\captionof{figure}{SADT A0}

\newpage
\parindent=15pt

Comme on peut le voir sur le SADT A0, nous avons découpé notre objectif en trois parties.

Dans un premier temps il faut capter les signaux. Pour cela il faut réaliser un balayage sur le radiogoniomètre pour détecter les bons signaux.

Ensuite, il faut analyser les signaux reçus pour s'assurer que nous sommes bien en présence d'un drone.

Enfin, il faut récupérer les données des radiogoniomètres pour déterminer la position du drone.


\section{FAST}

\hspace{-1.5cm}
\includegraphics[width=1.2\textwidth]{FAST.pdf}
\captionof{figure}{Diagramme FAST}


\section{Diagramme 3 axes}

\hspace{-1.5cm}
\includegraphics[width=1.2\textwidth]{3axes.pdf}
\captionof{figure}{Diagramme 3 axes}
\parindent=15pt
~\\

Le diagramme 3 axes ci-dessus présente les étapes clefs du traitement du problème. En effet, la
détection d’un drone nécessite de repérer une perturbation dans la bande de fréquence que l’on écoute,
de détecter la direction de laquelle elle provient et enfin de regrouper les données pour, à partir des
directions, obtenir la position.

\section{Fonctionnement de notre système}

Nous avons donc imaginé positionner plusieurs radiogoniomètres, chaque appareil indiquerait la direction du drone par rapport à sa position. Chacun d'eux serait connecté à un ordinateur central qui analyserai chacune des positions données par les radiogoniomètres et en déduirait la position du drone dans l'espace.

~\\

\includegraphics[width=0.8\textwidth]{SMART_logic}
\captionof{figure}{Schéma Logique du système}
\parindent=15pt



%%% Local Variables: 
%%% mode: latex
%%% TeX-master: "rapport_analyse"
%%% End: 


\part{État de l'art}
\chapter{Radio-goniométrie}

\section{Principe}

\subsection{Système Doppler}

\subsection{Système TDOA}

\subsection{Système Homing}








%%% Local Variables: 
%%% mode: latex
%%% TeX-master: "rapport_analyse"
%%% End:



%%%% CONCLUSION %%%%%%%%%

\chapter*{Conclusion}
\addcontentsline{toc}{chapter}{Conclusion}
Bien que sommaire, cette première analyse comprenant de la recherche bibliographique, de la veille technologique et de l'analyse fonctionnelle, nous permet de nous recentrer sur l'essentiel. Le domaine de la localisation de drone étant en plein essor, il est primordial de se concentrer sur un type de détection et d'avancer pas à pas.

Nous allons donc, dès à présent, nous attacher à la compréhension de la radiogoniométrie ainsi qu'à poursuivre la veille technologique afin de retenir les bonnes solutions de détection.

\newpage

%%%% ANNEXE %%%%%%%%%%%%

\part*{Annexe}
\appendix
\nocite{*}
\input{Partie_Organisation}
\chapter{Le Montréal 3V2}
\label{montreal}

Nous allons ici présenter la solution sur laquelle nous nous appuyons pour réaliser notre propre radiogoniomètre à effet Doppler, le Montréal 3V2.
Pour réaliser cette documentation nous nous sommes appuyé sur la documentation trouvé sur le site f1lvt \cite{montreal}

\section{Évolution du Montréal}

\includegraphics[width=\textwidth]{evolution}
\captionof{figure}{Evolution du Montréal}

\section{Avantages du Montréal 3v2}

\begin{center}
  \includegraphics[width=0.5\textwidth]{montreal}
  \captionof{figure}{Photographie prise du Montréal 3v2}
\end{center}

\parindent=15pt
Le Montréal 3v2 sert principalement à l'FNRASEC\footnote{Fédération Nationale des Radioamateurs au service de la Sécurité Civile, agrée de sécurité civile} et aux chasseurs d'onde amateurs. Ce radiogoniomètre est utilisé pour la détection de balise de détresse de 406MHz.

%Parmi ses avantages, on peut noter qu'il est facile à construire, son prix , et il est simple d'utilisation. 

Un des intérêts majeurs du Montréal 3-V2, c'est sa capacité de localiser des signaux très courts, son prix de revient est très raisonnable,son traitement très rapide et la mise en mémoire automatique du dernier relevé. On peu aussi noter qu'il est simple d'utilisation grâce a son affichage à 36LED disposé en cercle et qui indique la direction. De plus une LED centrale est indique le fonctionnement; verte la direction affichée est bonne, rouge le signal est insuffisant, la direction reste alors figée dans la dernière bonne direction reçue.

%on peut noter son affichage à 36 LED qui indique la direction de manière clair et efficace, et un réglage facilité par son écran LCD ou encore son filtre à capa commutée à très faible largeur de bande (0,5 Hz).

\section{Caractéristiques}

Le Montréal 3v2 est un radiogoniomètre à effet Doppler, il possède donc toutes les caractéristiques associé a ce type de radiogoniomètre.
~\\

\begin{tabular}{ l l l}
Fréquences & distance & moyenne portée\\
 & gamme & 50MHz-1.3GHz\\
 & démodulation & FM\\
Affichage & LED & 36LED\\
& écran & LCD en 2 lignes\\
Filtre & capa & très faible largeur de bande (0.5Hz)\\
Coût & & estimé à 50\euro \\
\end{tabular}


\section{Fonctionnement}
La partie centrale contient les circuits d'amplification et de commutation. Les 4 brins verticaux (les brins actifs) se fixent par BNC.

Les antennes sont alimentées de façon séquentielle pour imiter une antenne en rotation. Une fois que les antennes ont capté les ondes provenant du drone, il faut faire une démodulation et enlever tous les bruits.


Un système à LED permet de visualiser la composante continue qui passe dans les antennes. A partir du boîtier Doppler et de son menu de test, on peut ainsi vérifier individuellement chaque antenne. Ceci permet soit de faire fonctionner le système Doppler avec une antenne sur 4 %(fonctionnement conforme à la théorie avec une seule antenne tournante)
, soit avec 3 antennes sur 4. %(ce qui inverse le signal Doppler à 500 Hz ; mais ça fonctionne aussi bien voire mieux).


Trois microcontrôleurs Pics sont utilisés un 16F628A pour l'affichage, un 16F877A pour le circuit principal et un 12F675 comme diviseur de fréquence.

Ce Doppler est la version la plus récente et la plus performante de la série. Il commute les antennes et il affiche la direction mesurée sur la boussole à 36 LED. 


\section{Schéma bloc}

\includegraphics[width=\textwidth]{schemaBloc}
\captionof{figure}{Schéma bloc du Montréal 3v2}

\section{Liste des composants}
Voici la liste des composants pour la construction du Montréal 3v2:

\begin{tabular}{ l l l}

IC30&          LM386N-4&                  Ampli BF\\
IC50&          MAX267BCNG&          Filtre\\
IC51& PIC  12F675-I/P& PIC \\
IC52&          74HC4051N&               Filtre\\
IC53&          MAX492CPA &            Ampli Op\\
IC70& PIC  18F4520-I/P& PIC\\
VR20 &       7805 TO-220  &            Régulateur\\
X70&           20 MHz  HC49&           Quartz\\
D50&           1N5819       &                Diode Schottky\\
LCD20&      LCD 2X16,&                 Afficheur 2 lignes de 16 car.\\
IC1& PIC16F628A-I/P& PIC\\
LED1 - LED36& ø3mm, Rouge et/ou Vert&\\
LED37&                        3 ou 5mm Bicolore Rouge/Verte &\\
FB1 - FB8&                   Ferrites\footnote{+ composants passifs : Résistances, Condensateurs.}&\\
IC100&        = MAX232ACPE&        en option\\
Q100 &        = 2N2222 TO-92&\\
\end{tabular}



%%% Local Variables: 
%%% mode: latex
%%% TeX-master: "rapport_analyse"
%%% End: 


\newpage
\listoffigures
\printindex
\bibliographystyle{plain}
\bibliography{biblio}

\end{document}
%%%%%%%%%%%%%%%%% FIN DU DOCUMENT
